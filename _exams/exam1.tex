% Exam Template for UMTYMP and Math Department courses
%
% Using Philip Hirschhorn's exam.cls: http://www-math.mit.edu/~psh/#ExamCls
%
% run pdflatex on a finished exam at least three times to do the grading table on front page.
%
%%%%%%%%%%%%%%%%%%%%%%%%%%%%%%%%%%%%%%%%%%%%%%%%%%%%%%%%%%%%%%%%%%%%%%%%%%%%%%%%%%%%%%%%%%%%%%

% These lines can probably stay unchanged, although you can remove the last
% two packages if you're not making pictures with tikz.
\documentclass[11pt]{exam}
\RequirePackage{amssymb, amsfonts, amsmath, latexsym, verbatim, xspace, setspace}
\RequirePackage{tikz, pgflibraryplotmarks}

% By default LaTeX uses large margins.  This doesn't work well on exams; problems
% end up in the "middle" of the page, reducing the amount of space for students
% to work on them.
\usepackage[margin=1in]{geometry}
\usepackage{enumerate}
\usepackage{amsthm}

\theoremstyle{definition}
\newtheorem{soln}{Solution}

% Here's where you edit the Class, Exam, Date, etc.
\newcommand{\class}{Math 350 Section 1}
\newcommand{\term}{Fall 2024}
\newcommand{\examnum}{Exam I}
\newcommand{\examdate}{September 25, 2024}
\newcommand{\timelimit}{1 Hour 50 Minutes}
\newcommand{\ol}[1]{\overline{#1}}

% For an exam, single spacing is most appropriate
\singlespacing
% \onehalfspacing
% \doublespacing

% For an exam, we generally want to turn off paragraph indentation
\parindent 0ex

\begin{document} 

% These commands set up the running header on the top of the exam pages
\pagestyle{head}
\firstpageheader{}{}{}
\runningheader{\class}{\examnum\ - Page \thepage\ of \numpages}{\examdate}
\runningheadrule

\begin{flushright}
\begin{tabular}{p{2.8in} r l}
\textbf{\class} & \textbf{Name (Print):} & \makebox[2in]{\hrulefill}\\
\textbf{\term} &&\\
\textbf{\examnum} & \textbf{Student ID:}&\makebox[2in]{\hrulefill}\\
\textbf{\examdate} &&\\
\textbf{Time Limit: \timelimit} % & Teaching Assistant & \makebox[2in]{\hrulefill}
\end{tabular}\\
\end{flushright}
\rule[1ex]{\textwidth}{.1pt}


This exam contains \numpages\ pages (including this cover page) and
\numquestions\ problems.  Check to see if any pages are missing.  Enter
all requested information on the top of this page, and put your initials
on the top of every page, in case the pages become separated.\\

You may \textit{not} use your books or notes on this exam.

You are required to show your work on each problem on this exam.  The following rules apply:\\

\begin{minipage}[t]{3.7in}
\vspace{0pt}
\begin{itemize}

%\item \textbf{If you use a ``fundamental theorem'' you must indicate this} and explain
%why the theorem may be applied.

\item \textbf{Organize your work}, in a reasonably neat and coherent way, in
the space provided. Work scattered all over the page without a clear ordering will 
receive very little credit.  

\item \textbf{Mysterious or unsupported answers will not receive full
credit}.  A correct answer, unsupported by calculations, explanation,
or algebraic work will receive no credit; an incorrect answer supported
by substantially correct calculations and explanations might still receive
partial credit.  This especially applies to limit calculations.

\item If you need more space, use the back of the pages; clearly indicate when you have done this.

\item \textbf{Box Your Answer} where appropriate, in order to clearly indicate what you consider the answer to the question to be.
\end{itemize}

Do not write in the table to the right.
\end{minipage}
\hfill
\begin{minipage}[t]{2.3in}
\vspace{0pt}
%\cellwidth{3em}
\gradetablestretch{2}
\vqword{Problem}
\addpoints % required here by exam.cls, even though questions haven't started yet.	
\gradetable[v]%[pages]  % Use [pages] to have grading table by page instead of question

\end{minipage}
\newpage % End of cover page

%%%%%%%%%%%%%%%%%%%%%%%%%%%%%%%%%%%%%%%%%%%%%%%%%%%%%%%%%%%%%%%%%%%%%%%%%%%%%%%%%%%%%
%
% See http://www-math.mit.edu/~psh/#ExamCls for full documentation, but the questions
% below give an idea of how to write questions [with parts] and have the points
% tracked automatically on the cover page.
%

%%%%%%%%%%%%%%%%%%%%%%%%%%%%%%%%%%%%%%%%%%%%%%%%%%%%%%%%%%%%%%%%%%%%%%%%%%%%%%%%%%%%%

\begin{questions}

\addpoints

\question[10]\mbox{}
\textbf{TRUE or FALSE!}  Write  TRUE if the statement is true.  Otherwise, write FALSE.  Your response should be in ALL CAPS.  No justification is required.
\begin{enumerate}[(a)]
\item  
If $a$ and $b$ are nonzero numbers and $a < b$, then $1/a > 1/b$
\vspace{1.2in}
\item
Any non-empty set of positive integers has a minimum value
\vspace{1.2in}
\item
The relation $\mathcal R$ on $\mathbb R$ defined by $\mathcal R = \{(x,x): x\in \mathbb{R}\}$ is an equivalence relation
\vspace{1.2in}
\item
If $A\subseteq B$ then $\sup(A)\leq\sup(B)$
\vspace{1.2in}
\item  
If $A$ is a non-empty set of real numbers with $0 < x$ for all $x\in A$, then $0 < \sup(A)$
\vspace{1.2in}
\end{enumerate}

\newpage
\question[10]\mbox{}

\begin{enumerate}[(a)]
\item
Write down the definition of the upper bound of a set $A$.
\vspace{1in}
\item
Prove that the set of positive integers $\mathbb{Z}_+$ is not bounded above.
\end{enumerate}

\newpage
\question[10]\mbox{}
\begin{enumerate}[(a)]
\item Write down the definition of the infimum of a set $A$
\vspace{1in}

\item Determine the infimum of the set 

$$A = \left\lbrace\frac{1}{n}: n\in\mathbb{N}\right\rbrace.$$
\vspace{1.3in}

\item Prove your answer for part (b).
\end{enumerate}

\newpage
\question[10]\mbox{}

\begin{enumerate}[(a)]
\item  Write down the four order axioms of the real numbers (ie. A6-A9).
\vspace{4in}
\item  Use the order axioms to prove that if $x$, $y$, and $z$ are real numbers with $x < y$ and $z > 0$, then $xz < yz$.  Carefully state each time you use one of the axioms.
\end{enumerate}

\newpage
\question[10]\mbox{}

Let $A$ and $B$ be sets and suppose $f: A\rightarrow B$ is a function.

\begin{enumerate}[(a)]
\item  Write down the definition of $f$ being injective.
\vspace{1in}
\item  Write down the definition of $f$ being surjective.
\vspace{1in}
\item  Write down the definition of $f$ being bijective.
\vspace{1in}
\item  Prove that the function $f: \mathbb Z_+\rightarrow\mathbb Z$ defined by

$$f(x) = \left\lbrace\begin{array}{cc}
x/2, & x\ \text{is even}\\
(1-x)/2, & x\ \text{is odd}
\end{array}\right.$$

is injective and surjective.
\end{enumerate}

\newpage
\question[10]\mbox{}

Suppose that $A$ and $B$ are two non-empty sets of real numbers, both of which are bounded above, and let $C=\{a+b: a\in A,\ b\in B\}$.

\begin{enumerate}[(a)]
\item State the Completeness Axiom
\vspace{1in}
\item Prove that $C$ is bounded above
\vspace{3in}
\item Prove that $\sup(C)\leq \sup(A) + \sup(B)$
\end{enumerate}

\newpage
\question[10]\mbox{}
Consider the family of sets $\{A_i: i\in I\}$ where the index set $I = \mathbb{Z}$ and $A_i = (i,i+1)$.
\begin{enumerate}[(a)]
\item Determine the value of $\bigcap_{i\in I} A_i$
\vspace{1in}
\item Carefully prove your answer in part (a) is correct.
\vspace{3in}
\item Determine the value of $\bigcup_{i\in I} A_i$
\vspace{1in}
\item Carefully prove your answer in part (c) is correct.
\end{enumerate}

\end{questions}

\end{document}

