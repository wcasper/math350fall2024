% !TEX TS-program = pdflatex
% !TEX encoding = UTF-8 Unicode

% This file is a template using the "beamer" package to create slides for a talk or presentation
% - Giving a talk on some subject.
% - The talk is between 15min and 45min long.
% - Style is ornate.

% MODIFIED by Jonathan Kew, 2008-07-06
% The header comments and encoding in this file were modified for inclusion with TeXworks.
% The content is otherwise unchanged from the original distributed with the beamer package.

\documentclass{beamer}
\setbeamercovered{invisible}



% Copyright 2004 by Till Tantau <tantau@users.sourceforge.net>.
%
% In principle, this file can be redistributed and/or modified under
% the terms of the GNU Public License, version 2.
%
% However, this file is supposed to be a template to be modified
% for your own needs. For this reason, if you use this file as a
% template and not specifically distribute it as part of a another
% package/program, I grant the extra permission to freely copy and
% modify this file as you see fit and even to delete this copyright
% notice. 


\mode<presentation>
{
  \usetheme{Warsaw}
  % or ...

  % or whatever (possibly just delete it)
}


\usepackage[english]{babel}
% or whatever

\usepackage[utf8]{inputenc}
% or whatever

\usepackage{times}
\usepackage[T1]{fontenc}
% Or whatever. Note that the encoding and the font should match. If T1
% does not look nice, try deleting the line with the fontenc.

%%% MATH RELATED


%%% AMS math stuff
\usepackage{amsmath}
\usepackage{amssymb}
\usepackage{amsthm}
\usepackage{mathrsfs}
\usepackage{enumerate}

%%% Theorem environments
\newtheorem{thm}{Theorem}[subsection]
\newtheorem{prop}[thm]{Proposition}
\newtheorem{cor}{Corollary}[thm]
\newtheorem{por}[thm]{Porism}
\newtheorem{lem}[thm]{Lemma}
\theoremstyle{definition}
\newtheorem{prob}[thm]{Problem}
\newtheorem{soln}{Solution}
\newtheorem{defn}[thm]{Definition}
\newtheorem{ex}[thm]{Example}
\newtheorem{quest}[thm]{Question}
\newtheorem{remk}[thm]{Remark}

%%% Typesetting shortcuts
\newcommand{\tn}[1]{\textnormal{#1}}
\newcommand{\ol}[1]{\overline{#1}}
\newcommand{\wt}[1]{\widetilde{#1}}
\newcommand{\wh}[1]{\widehat{#1}}
\newcommand{\vocab}[1]{\textbf{#1}\index{#1}}

%%% Math shortcuts
\newcommand{\bbr}{\mathbb R}
\newcommand{\bbz}{\mathbb Z}
\newcommand{\bbq}{\mathbb Q}
\newcommand{\bbn}{\mathbb N}
\newcommand{\bbf}{\mathbb F}
\newcommand{\bbc}{\mathbb C}
\newcommand{\bbd}{\mathbb D}
\newcommand{\bba}{\mathbb A}
\newcommand{\bbp}{\mathbb P}
\newcommand{\bbg}{\mathbb G}
\newcommand{\bbv}{\mathbb V}
\newcommand{\dih}[1]{\mathcal D_{#1}}
\newcommand{\sym}[1]{\mathcal S_{#1}}
\newcommand{\vspan}{\tn{span}}
\newcommand{\trace}{\tn{trace}}
\newcommand{\diff}{\backslash}
\newcommand{\stab}{\tn{stab}}
\newcommand{\conv}{\tn{conv}}
\newcommand{\img}{\tn{img}}
\newcommand{\coker}{\tn{coker}}
\newcommand{\id}{\tn{id}}
\newcommand{\Hom}{\tn{Hom}}
\newcommand{\End}{\tn{End}}
\newcommand{\Aut}{\tn{Aut}}
\newcommand{\aut}{\tn{Aut}}
\newcommand{\ann}{\tn{Ann}}
\newcommand{\GL}{\tn{GL}}
\newcommand{\Gr}{\tn{Gr}}
\newcommand{\lord}{\preccurlyeq}
\newcommand{\rord}{\succcurlyeq}
\newcommand{\tr}{\textnormal{Tr}}
\newcommand{\Tr}{\textnormal{Tr}}
\newcommand{\bbl}{\mathbb{L}}
\newcommand{\C}{\mathscr{C}}
\newcommand{\X}{\mathscr{X}}
\renewcommand{\S}{\mathscr{S}}
\newcommand{\M}{\mathscr{M}}
\renewcommand{\L}{\mathcal{L}}


%%% Algebraic Geometry
\newcommand{\height}{\textnormal{ht}}
\newcommand{\A}{\mathbb{A}}
\newcommand{\p}{\mathfrak{p}}
\newcommand{\sheaf}[1]{\mathcal{#1}}
\newcommand{\spec}{\textnormal{Spec}}
\newcommand{\proj}{\textnormal{Proj}}
\newcommand{\Aff}{\textnormal{Aff}}
\newcommand{\skel}{\textnormal{skel}}
\newcommand{\supp}{\textnormal{supp}}
\newcommand{\orb}{\textnormal{orb}}
\newcommand{\Proj}{\textnormal{Proj}}
\newcommand{\Pic}{\textnormal{Pic}}
\newcommand{\Rees}{\textnormal{Rees}}
\newcommand{\shom}{\mathcal{H}om}

\renewcommand*\arraystretch{1.3}

%%% paper specific definitions
\newcommand{\weyl}{\Omega}
\newcommand{\weyll}{{\widehat{\Omega}}}
\newcommand{\weylll}{{\widetilde{\Omega}}}
\newcommand{\mweyl}{{M_N(\Omega)}}
\newcommand{\mweyll}{{M_N(\widehat{\Omega})}}
\newcommand{\mweylll}{{M_N(\widetilde{\Omega})}}
\newcommand{\seq}{\text{Seq}}
\newcommand{\tail}{\text{Tail}}
\newcommand{\Ad}{\textnormal{Ad}}
\newcommand{\sech}{\textnormal{sech}}
\newcommand{\colim}{\varinjlim}
\newcommand{\limit}{\varprojlim}
\newcommand{\Bis}{\textnormal{Bis}}
\newcommand{\m}{\mathfrak{m}}
\newcommand{\mxx}[4]{\left(\begin{array}{cc} #1 & #2\\ #3 & #4 \end{array}\right)}
\newcommand{\diag}{\text{diag}}
\newcommand{\qdet}{\textnormal{qdet}}
\newcommand{\mdet}{\textnormal{mdet}}
\newcommand{\mtau}{\mathcal{T}}
\newcommand{\cof}{\textnormal{cof}}
\newcommand{\minor}{\textnormal{minor}}
\newcommand{\holo}{Holo}
\newcommand{\ord}{\textnormal{order}}
\newcommand{\mult}{\mathfrak M}






\title{MATH 350-2 Advanced Calculus} 
\subtitle
{} % (optional)

\author[W.R. Casper] % (optional, use only with lots of authors)
{W.R. Casper}
% - Use the \inst{?} command only if the authors have different
%   affiliation.

\institute[California State University Fullerton] % (optional, but mostly needed)
{
  Department of Mathematics\\
  California State University Fullerton}
% - Use the \inst command only if there are several affiliations.
% - Keep it simple, no one is interested in your street address.

\subject{Talks}
% This is only inserted into the PDF information catalog. Can be left
% out. 



% If you have a file called "university-logo-filename.xxx", where xxx
% is a graphic format that can be processed by latex or pdflatex,
% resp., then you can add a logo as follows:

% \pgfdeclareimage[height=0.5cm]{university-logo}{university-logo-filename}
% \logo{\pgfuseimage{university-logo}}



% Delete this, if you do not want the table of contents to pop up at
% the beginning of each subsection:
\AtBeginSubsection[]
{
  \begin{frame}<beamer>{Outline}
    \tableofcontents[currentsection,currentsubsection]
  \end{frame}
}


% If you wish to uncover everything in a step-wise fashion, uncomment
% the following command: 

%\beamerdefaultoverlayspecification{<+->}


\begin{document}

\begin{frame}
  \titlepage
\end{frame}

\begin{frame}{Outline}
  \tableofcontents
  % You might wish to add the option [pausesections]
\end{frame}

% Since this a solution template for a generic talk, very little can
% be said about how it should be structured. However, the talk length
% of between 15min and 45min and the theme suggest that you stick to
% the following rules:  

% - Exactly two or three sections (other than the summary).
% - At *most* three subsections per section.
% - Talk about 30s to 2min per frame. So there should be between about
%   15 and 30 frames, all told.

\section{Real Analysis Lecture 2}

\subsection{Irrational numbers}
\begin{frame}{Types of numbers}
\begin{center}
\includegraphics[width=0.8\linewidth]{fig/numbers}
\end{center}
\end{frame}

\begin{frame}{Irrational numbers}
Numbers which are not of the form $a/b$ with $a,b\in\mathbb{Z}$ are called \vocab{irrational}.\\
\begin{itemize}
\item Discovered by Hippasus, a pythagorean (a student of Pythagoras)
\end{itemize}
\begin{center}
\includegraphics[width=0.4\linewidth]{fig/pentagram}
\end{center}
\end{frame}

\begin{frame}{Irrational numbers}
Numbers which are not of the form $a/b$ with $a,b\in\mathbb{Z}$ are called \vocab{irrational}.\\
\begin{itemize}
\item Pythagoras drowned Hippasus at sea for his discovery
\pause
\item dramatic reenactment, found online:
\pause
\end{itemize}
\pause
\begin{center}
\includegraphics[width=0.6\linewidth]{fig/hippasus-death}
\end{center}
\end{frame}

\begin{frame}{Irrational numbers}
Numbers which are not of the form $a/b$ with $a,b\in\mathbb{Z}$ are called \vocab{irrational}.\\
\begin{itemize}
\item Pythagoras drowned Hippasus at sea for his discovery
\item anime style:
\pause
\end{itemize}
\begin{center}
\includegraphics[width=0.4\linewidth]{fig/jojo-hippasus}
\end{center}
\end{frame}

\begin{frame}{Challenge!}
\begin{prob}
Can you prove that if $n\in\mathbb{Z}_+$ is not a perfect square, then $\sqrt{n}$ is irrationial?
(Apostol Theorem 1.10)
\end{prob}
\pause
\begin{soln}
We prove by contradiction. \pause
Assume $\sqrt{n} = a/b$ for integers $a,b$. \pause
Without loss of generality, we may assume $\gcd(a,b) = 1$. \pause
Then $b^2n = a^2$. \pause
Since $\gcd(a,b) = 1$, $a^2$ must divide $n$, ie. $n=n'a^2$. \pause
It follows that $b^2n' = 1$, so $b^2$ divides $1$. \pause
This implies $n'$ divides $1$, so $n'=\pm 1$. \pause
Since $n>0$, $n'>0$ so $n'=1$. \pause
Thus $n=a^2$, which is a contradiction.
\end{soln}
\end{frame}

\begin{frame}{Algebraic and Transcendental}
Irrational numbers can be further divided into to categories
\pause
\begin{enumerate}
\item \textbf{algebraic numbers:} numbers which are roots of polynomials with integer coefficients, like
$$\sqrt{2},\quad\sqrt{3},\quad\text{and}\quad\sqrt[3]{\sqrt{5}  + \sqrt{7}}.$$
\pause
\item \textbf{transcendental numbers:} numbers which are not algebraic, like
$$\pi,\quad e,\quad e^\pi,\quad\text{and maybe}\quad\pi+e?$$
\pause
\item transcendentals are mysterious ... but most real numbers are transcendental!
\end{enumerate}
\end{frame}

\subsection{Upper Bound and Supremum}

\begin{frame}{Upper bounds}
An \textbf{upper bound} for a set $S\subseteq\mathbb{R}$ is a number $b$ such that
$$x \leq b\quad\text{for all}\ x\in S.$$
\pause
In this case, we say $S$ is \textbf{bounded above} by $b$.\\
\pause
If $b\in S$ also, then $b$ is called a \textbf{maximal element} of $S$
\pause
\textbf{Examples:}
\begin{itemize}
\pause
\item $34$ is an upper bound of $[-1,5]$, but not a maximal element
\pause
\item $5$ is a maximal element of $[-1,5]$
\pause
\item $3$ is an upper bound of $[0,3)$, but not a maximal element
\pause
\item $\mathbb{Z}_+$ has no upper bound
\pause
\item $[3,7)$ has an upper bound but no maximal element
\end{itemize}
\end{frame}


\begin{frame}{Challenge!}
\begin{prob}
Show that if $S$ has a maximal element, then it is unique.
\end{prob}
\pause
\textbf{Hint: use the definition!}
\pause
\begin{soln}
Suppose that $b_1$ and $b_2$ are both maximal elements of $S$.\\
\pause
Then $x\leq b_1$ and $x\leq b_2$ for all $x\in S$.\\
\pause
Moreover, $b_1\in S$ and $b_2\in S$.\\
\pause
Since $b_1\in S$ and $b_2$ is an upper bound of $S$, $b_1\leq b_2$.\\
\pause
Likewise, since $b_2\in S$ and $b_1$ is an upper bound of $S$, $b_2\leq b_1$.\\
\pause
By the trichotomy, we find $b_1=b_2$.
\end{soln}
\pause
Now we can say \emph{the} maximum, $\max(S)$
\end{frame}

\begin{frame}{Supremum}
A \textbf{supremum} of a set $S$ of real numbers is a real number $b\in\mathbb{R}$ such that
\begin{itemize}
\item $b$ is an upper bound of $S$
\item if $b'<b$, then $b'$ is not an upper bound of $S$
\end{itemize}
\pause
In other words
$$\text{a supremum is a \bf{least upper bound}}$$
\end{frame}

\begin{frame}{Challenge!}
\begin{prob}
Show that if $S$ has a supremum, then it is unique.
\end{prob}
\pause
\textbf{Hint: use the definition!}
\pause
\begin{soln}
Suppose that $b_1$ and $b_2$ are both suprema of $S$.\\
\pause
Then $b_1$ and $b_2$ are both upper bounds of $S$.\\
\pause
Since $b_1$ is a least upper bound, $b_1\leq b_2$.\\
\pause
Since $b_2$ is also a least upper bound, $b_2\leq b_1$.\\
\pause
By the trichotomy, we find $b_1=b_2$.
\end{soln}
\pause
Now we can say \emph{the} supremum, $\sup(S)$
\end{frame}

\begin{frame}{Challenge!}
\begin{prob}
Show that if $S$ has a maximum element.  Then $S$ has a supremum and $\max(S)=\sup(S)$
\end{prob}
\pause
\textbf{Hint: use the definition!}
\pause
\begin{soln}
Let $b=\max(S)$.\\
\pause
Then $b$ is an upper bound of $S$ and $b\in S$.\\
\pause
If $b'$ is another upper bound of $S$, then by definition $b\leq b'$.\\
\pause
Thus $b$ is the least upper bound of $S$.
\end{soln}
\end{frame}

\begin{frame}{Completeness Axiom}
Now we have the machinery in place to state the Completeness Axiom.
\pause
\begin{enumerate}[\text{A}1]
\setcounter{enumi}{9}
\pause
\item \textbf{completeness axiom}: if $S$ is any subset of real numbers which is bounded above, then it has a supremum $\sup(S)$
\end{enumerate}
\pause
\textbf{Examples:}
\begin{itemize}
\pause
\item $3$ is the supremum of $(0,3)$
\pause
\item $1$ is the supremum of
$$\left\lbrace
\frac{n}{n+1}: n\in\mathbb{Z}_+
\right\rbrace$$
\pause
\item $\pi$ is the supremum of
$$\{3,3.1,3.14,3.141,3.1415,3.14159,3.141592,\dots\}.$$
\end{itemize}
\end{frame}

\begin{frame}{Challenge}
\begin{prob}
Prove that if $A$ is a set of integers that is bounded above, then $A$ has a maximum.
\end{prob}
\end{frame}

\begin{frame}{Challenge}
\begin{soln}
The set $A$ is bounded above, so it has a supremum $b=\sup(A)$. \\
\pause
Since $b-1$ is not an upper bound, so there exists $a\in A$ with $b-1 < a$. \\
\pause
If $a'\in A$, then $a'$ is an integer and therefore $a' = a+k$ for $k\in \mathbb{Z}$. \\
\pause
Since $b$ is an upper bound, $b\geq a+k>b-1+k$, making $0 > k-1$. \\
\pause
Therefore $k\leq 0$ and $a'\leq a$. \\
\pause
It follows that $a$ is an upper bound of $A$, and since $a\in A$ it is a maximum.
\end{soln}
\end{frame}

\begin{frame}{Lower bounds and infima}
A \textbf{lower bound} for a set $S\subseteq\mathbb{R}$ is a number $b$ such that
$$b \leq x\quad\text{for all}\ x\in S.$$
\pause
In this case, we say $S$ is \textbf{bounded below} by $b$.\\
\pause
If $b\in S$ also, then $b$ is called a \textbf{minimal element} of $S$.
\pause
An \textbf{infimum} of a set $S$ of real numbers is a real number $b\in\mathbb{R}$ such that
\begin{itemize}
\item $b$ is a lower bound of $S$
\item if $b<b'$, then $b'$ is not a lower bound of $S$
\end{itemize}
\pause
In other words
$$\text{an infimum is a \bf{greatest lower bound}}$$
\end{frame}

\begin{frame}{Properties of Suprema}
The first property of suprema is that they must be arbitrarily close to elements of the set.
\begin{thm}[Approximation Property]
Let $S\subseteq\mathbb{R}$ be bounded above, and let $b = \sup(S)$.
Then for all $\epsilon > 0$, there exists $a\in S$ with
$$b-\epsilon < a < b.$$
\end{thm}
\end{frame}
\begin{frame}{Properties of Suprema}
\begin{proof}
Since $b-\epsilon < b$, the definition of a supremum implies $b-\epsilon$ cannot be an upper bound.\\
\pause
Therefore thre must exist $a\in S$ with $a > b-\epsilon$.
\end{proof}
\end{frame}

\begin{frame}{Important example: The number $e$}
For each $n$, let $s_n$ denote the sum
$$s_n = 1 + \frac{1}{1!} + \frac{1}{2!} + \dots + \frac{1}{n!}.$$
\pause
Notice that
$$s_1 < s_2 < s_3 < s_4 < \dots$$
\pause
Consider the set
$$S = \{ s_n: n\in\mathbb{Z}_+\}.$$
\pause
For every $n$, 
$$s_n \leq 1 + 1 + \frac{1}{2} + \frac{1}{2^2} + \frac{1}{2^3} + \dots + \frac{1}{2^n} = 3 - \frac{1}{2^{n}} < 3$$
so $S$ is bounded above by $3$.
\pause
Therefore $S$ has a supremum, $\sup(S)$.
\end{frame}

\begin{frame}{Important example: The number $e$}
Even if $\epsilon$ is very small ($\epsilon = 0.000000001)$, $\sup(S)-\epsilon$ is not an upper bound of $S$.\\
\pause
So there exists $N$ with $s_N > \sup(S)-\epsilon$.
\pause
$$s_1 < s_2 < s_3 < s_4 < \dots < \sup(S)-\epsilon < s_N < \dots < \sup(S).$$
\pause
This says $\sup(S)$ is \emph{really} close to $s_N$.
\pause
Taking $\epsilon$ smaller and smaller
\pause
$$\sup(S) = 1 + \frac{1}{1!} + \frac{1}{2!} + \frac{1}{3!} + \frac{1}{4!}  + \dots$$
\pause
If we remember Taylor series
\pause
$$e^x = 1 + \frac{x}{1!} + \frac{x^2}{2!} + \frac{x^3}{3!} + \frac{x^4}{4!}  + \dots$$
\pause
We've just shown that $e^1=e$ exists.
\end{frame}


\begin{frame}{Properties of Suprema}
Also, suprema play nicely with addition.
\begin{thm}[Additive Property]
Let $A,B\subseteq\mathbb{R}$ be bounded above set
$$C = \{x+y: x\in A,\ y\in B\}.$$
Then $C$ is bounded above and
$$\sup(C) = \sup(A) + \sup(B).$$
\end{thm}
\end{frame}
\begin{frame}{Properties of Suprema}
\begin{proof}
Let $a = \sup(A)$, $b = \sup(B)$, and $c=\sup(C)$. \\
\pause
We will prove $c \leq a+b$ and then $a+b\leq c$. \\
\pause
First, note $a$ is an upper bound of $A$, so $x\leq a$ for all $x\in A$. \\
\pause
Also $b$ is an upper bound of $b$, so $y\leq b$ for all $y\in B$. \\
\pause
If $z\in C$, then $z=x+y$ for some $x\in A$ and $y\in B$, and therefore $z = x+y \leq a+b$. \\
\pause
Therefore $a+b$ is an upper bound of $C$. \\
\pause
This means $c\leq a+b$. \\
\pause
Next, note for all $x\in A$ and $y\leq B$ that $x\leq c-y$. \\
\pause
Therefore $c-y$  is an upper bound of $A$ and $a\leq c-y$. \\
\pause
It follows that $y\leq c-a$ for all $y\in B$. \\
\pause
Thus $c-a$ is an upper bound of $B$, and it follows $b\leq c-a$. \\
\pause
Therefore $a+b\leq c$.
\end{proof}
\end{frame}


\begin{frame}{Challenge}
\begin{prob}
Use the completeness axiom to prove that the $\mathbb Z_+$ is not bounded above.
(Apostol Theorem 1.17)
\end{prob}
\pause
\textbf{Hint:} assume it is and consider $\sup(\mathbb Z_+)-1$
\pause
\begin{soln}
Assume it is.\\\pause
The completeness axiom implies that $b=\sup(\mathbb Z_+)$ exists\\\pause
The number $b$ is an upper bound and if $b' < b$, then $b'$ is not.\\\pause
Then $b-1 < b$ by Axiom 7, so $b-1$ cannot be an upper bound.\\\pause
This means there exists $n\in\mathbb{Z}_+$ with $b-1 < n$.\\\pause
It follows from Axiom 7 that $b < n+1$.\\\pause
However, $n+1\in\mathbb{Z}$, so this contradicts $b$ being an upper bound.
\end{soln}
\end{frame}

\begin{frame}{Archimedian property}
\begin{thm}[Apostol Theorem 1.18]
For every $x\in\mathbb{R}$, there exists $n\in\mathbb{Z}_+$ with $n>x$.
\end{thm}
\pause
\begin{proof}
If not, then $x$ is an upper bound of $\mathbb{Z}_+$.
\end{proof}
\pause
\begin{thm}[Archimedian Property of Reals]
For every $x,y\in\mathbb{R}$ with $x>0$, there exists $n\in\mathbb{Z}_+$ with $y<nx$.
\end{thm}
\pause
\begin{proof}
Replace $x$ with $y/x$ in the previous theorem.
\end{proof}
\end{frame}

\subsection{Decimal expansions}

\begin{frame}{Decimal approximations}
A \textbf{finite decimal expansion} is an expression
$$r = a_0 + \frac{a_1}{10} + \frac{a_2}{100} + \frac{a_3}{10^3} + \dots + \frac{a_n}{10^n},$$
where $a_0\in\mathbb Z_+$ and $0\leq a_k\leq 9$ for $1\leq k\leq n$.\\
\pause
\textbf{Notation:}
$$a_0.a_1a_2a_3\dots a_n.$$
\pause
Any positive real number $x>0$ can be approximated by a finite decimal expansion.
\end{frame}

\begin{frame}{Decimal approximations}
\begin{thm}[Apostol Theorem 1.20]
For any real $x > 0$ and $n\in \mathbb{Z}_+$, there exists a finite decimal expansion $r_n=a_0.a_1a_2\dots a_n$ with
$$r_n \leq x < r_n + \frac{1}{10^n}.$$
\end{thm}
\end{frame}

\begin{frame}{Decimal approximations}
\begin{proof}
Consider the set
$$A = \{a\in\mathbb{Z}: a\leq x\}.$$
\pause
The set $A$ is a set of integers which is bounded above by $x$, so it has a maximum $a_0$. \\
\pause
Then clearly $x_1 = a-a_0\in [0,1)$. \\
\pause
Define $a_1,a_2,a_3,\dots$ and $x_1,x_2,x_3,\dots$ recursively by
$$a_k = \max\{a\in\mathbb{Z}_+: a\leq 10x_{k}\}$$
and $x_{k+1}=10x_k-a_k$.
\pause
Then $0\leq a_k \leq 9$ for all $k\geq 1$ 
\pause
and 
$$a_0 + \frac{a_1}{10} + \frac{a_2}{100} + \dots + \frac{a_n}{10^n}
\leq x < a_0 + \frac{a_1}{10} + \frac{a_2}{100} + \dots + \frac{a_n+1}{10^n}
$$
\end{proof}
\end{frame}

\begin{frame}{Decimal expansions}
We say that $x>0$ has the decimal expansion $a_0.a_1a_2a_3\dots$ and write
$$x = a_0.a_1a_2a_3a_4\dots$$
\pause
if for all $n\in\mathbb{Z}_+$,
$$a_0 + \frac{a_1}{10} + \frac{a_2}{100} + \dots + \frac{a_n}{10^n}
\leq x < a_0 + \frac{a_1}{10} + \frac{a_2}{100} + \dots + \frac{a_n+1}{10^n}.
$$
\pause
Note: this is slightly different than the usual limit meaning, for two good reasons:
\begin{enumerate}
\pause
\item we haven't defined limits
\pause
\item with this definition, numbers have unique decimal expansions!
\pause
$$1\neq 0.999999999\dots$$
\pause
$$1 \nless 0 + \frac{9}{10} + \frac{9}{100} + \dots + \frac{9+1}{10^n} = 1.$$
\end{enumerate}
\end{frame}

\end{document}


