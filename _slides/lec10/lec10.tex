% !TEX TS-program = pdflatex
% !TEX encoding = UTF-8 Unicode

% This file is a template using the "beamer" package to create slides for a talk or presentation
% - Giving a talk on some subject.
% - The talk is between 15min and 45min long.
% - Style is ornate.

% MODIFIED by Jonathan Kew, 2008-07-06
% The header comments and encoding in this file were modified for inclusion with TeXworks.
% The content is otherwise unchanged from the original distributed with the beamer package.

\documentclass{beamer}
\setbeamercovered{invisible}



% Copyright 2004 by Till Tantau <tantau@users.sourceforge.net>.
%
% In principle, this file can be redistributed and/or modified under
% the terms of the GNU Public License, version 2.
%
% However, this file is supposed to be a template to be modified
% for your own needs. For this reason, if you use this file as a
% template and not specifically distribute it as part of a another
% package/program, I grant the extra permission to freely copy and
% modify this file as you see fit and even to delete this copyright
% notice. 


\mode<presentation>
{
  \usetheme{Warsaw}
  % or ...

  % or whatever (possibly just delete it)
}


\usepackage[english]{babel}
% or whatever

\usepackage[utf8]{inputenc}
% or whatever

\usepackage{times}
\usepackage[T1]{fontenc}
% Or whatever. Note that the encoding and the font should match. If T1
% does not look nice, try deleting the line with the fontenc.

%%% MATH RELATED


%%% AMS math stuff
\usepackage{amsmath}
\usepackage{amssymb}
\usepackage{amsthm}
\usepackage{mathrsfs}
\usepackage{enumerate}

%%% Theorem environments
\newtheorem{thm}{Theorem}[subsection]
\newtheorem{prop}[thm]{Proposition}
\newtheorem{cor}{Corollary}[thm]
\newtheorem{por}[thm]{Porism}
\newtheorem{lem}[thm]{Lemma}
\theoremstyle{definition}
\newtheorem{prob}[thm]{Problem}
\newtheorem{soln}{Solution}
\newtheorem{defn}[thm]{Definition}
\newtheorem{ex}[thm]{Example}
\newtheorem{quest}[thm]{Question}
\newtheorem{remk}[thm]{Remark}

%%% Typesetting shortcuts
\newcommand{\tn}[1]{\textnormal{#1}}
\newcommand{\ol}[1]{\overline{#1}}
\newcommand{\wt}[1]{\widetilde{#1}}
\newcommand{\wh}[1]{\widehat{#1}}
\newcommand{\vocab}[1]{\textbf{#1}\index{#1}}

%%% Math shortcuts
\newcommand{\bbr}{\mathbb R}
\newcommand{\bbz}{\mathbb Z}
\newcommand{\bbq}{\mathbb Q}
\newcommand{\bbn}{\mathbb N}
\newcommand{\bbf}{\mathbb F}
\newcommand{\bbc}{\mathbb C}
\newcommand{\bbd}{\mathbb D}
\newcommand{\bba}{\mathbb A}
\newcommand{\bbp}{\mathbb P}
\newcommand{\bbg}{\mathbb G}
\newcommand{\bbv}{\mathbb V}
\newcommand{\dih}[1]{\mathcal D_{#1}}
\newcommand{\sym}[1]{\mathcal S_{#1}}
\newcommand{\vspan}{\tn{span}}
\newcommand{\trace}{\tn{trace}}
\newcommand{\diff}{\backslash}
\newcommand{\stab}{\tn{stab}}
\newcommand{\conv}{\tn{conv}}
\newcommand{\img}{\tn{img}}
\newcommand{\coker}{\tn{coker}}
\newcommand{\id}{\tn{id}}
\newcommand{\Hom}{\tn{Hom}}
\newcommand{\End}{\tn{End}}
\newcommand{\Aut}{\tn{Aut}}
\newcommand{\aut}{\tn{Aut}}
\newcommand{\ann}{\tn{Ann}}
\newcommand{\GL}{\tn{GL}}
\newcommand{\Gr}{\tn{Gr}}
\newcommand{\lord}{\preccurlyeq}
\newcommand{\rord}{\succcurlyeq}
\newcommand{\tr}{\textnormal{Tr}}
\newcommand{\Tr}{\textnormal{Tr}}
\newcommand{\bbl}{\mathbb{L}}
\newcommand{\C}{\mathscr{C}}
\newcommand{\X}{\mathscr{X}}
\renewcommand{\S}{\mathscr{S}}
\newcommand{\M}{\mathscr{M}}
\renewcommand{\L}{\mathcal{L}}


%%% Algebraic Geometry
\newcommand{\height}{\textnormal{ht}}
\newcommand{\A}{\mathbb{A}}
\newcommand{\p}{\mathfrak{p}}
\newcommand{\sheaf}[1]{\mathcal{#1}}
\newcommand{\spec}{\textnormal{Spec}}
\newcommand{\proj}{\textnormal{Proj}}
\newcommand{\Aff}{\textnormal{Aff}}
\newcommand{\skel}{\textnormal{skel}}
\newcommand{\supp}{\textnormal{supp}}
\newcommand{\orb}{\textnormal{orb}}
\newcommand{\Proj}{\textnormal{Proj}}
\newcommand{\Pic}{\textnormal{Pic}}
\newcommand{\Rees}{\textnormal{Rees}}
\newcommand{\shom}{\mathcal{H}om}

\renewcommand*\arraystretch{1.3}

%%% paper specific definitions
\newcommand{\weyl}{\Omega}
\newcommand{\weyll}{{\widehat{\Omega}}}
\newcommand{\weylll}{{\widetilde{\Omega}}}
\newcommand{\mweyl}{{M_N(\Omega)}}
\newcommand{\mweyll}{{M_N(\widehat{\Omega})}}
\newcommand{\mweylll}{{M_N(\widetilde{\Omega})}}
\newcommand{\seq}{\text{Seq}}
\newcommand{\tail}{\text{Tail}}
\newcommand{\Ad}{\textnormal{Ad}}
\newcommand{\sech}{\textnormal{sech}}
\newcommand{\colim}{\varinjlim}
\newcommand{\limit}{\varprojlim}
\newcommand{\Bis}{\textnormal{Bis}}
\newcommand{\m}{\mathfrak{m}}
\newcommand{\mxx}[4]{\left(\begin{array}{cc} #1 & #2\\ #3 & #4 \end{array}\right)}
\newcommand{\diag}{\text{diag}}
\newcommand{\qdet}{\textnormal{qdet}}
\newcommand{\mdet}{\textnormal{mdet}}
\newcommand{\mtau}{\mathcal{T}}
\newcommand{\cof}{\textnormal{cof}}
\newcommand{\minor}{\textnormal{minor}}
\newcommand{\holo}{Holo}
\newcommand{\ord}{\textnormal{order}}
\newcommand{\mult}{\mathfrak M}






\title{MATH 350-2 Advanced Calculus} 
\subtitle
{} % (optional)

\author[W.R. Casper] % (optional, use only with lots of authors)
{W.R. Casper}
% - Use the \inst{?} command only if the authors have different
%   affiliation.

\institute[California State University Fullerton] % (optional, but mostly needed)
{
  Department of Mathematics\\
  California State University Fullerton}
% - Use the \inst command only if there are several affiliations.
% - Keep it simple, no one is interested in your street address.

\subject{Talks}
% This is only inserted into the PDF information catalog. Can be left
% out. 



% If you have a file called "university-logo-filename.xxx", where xxx
% is a graphic format that can be processed by latex or pdflatex,
% resp., then you can add a logo as follows:

% \pgfdeclareimage[height=0.5cm]{university-logo}{university-logo-filename}
% \logo{\pgfuseimage{university-logo}}



% Delete this, if you do not want the table of contents to pop up at
% the beginning of each subsection:
\AtBeginSubsection[]
{
  \begin{frame}<beamer>{Outline}
    \tableofcontents[currentsection,currentsubsection]
  \end{frame}
}


% If you wish to uncover everything in a step-wise fashion, uncomment
% the following command: 

%\beamerdefaultoverlayspecification{<+->}


\begin{document}

\begin{frame}
  \titlepage
\end{frame}

\begin{frame}{Outline}
  \tableofcontents
  % You might wish to add the option [pausesections]
\end{frame}

% Since this a solution template for a generic talk, very little can
% be said about how it should be structured. However, the talk length
% of between 15min and 45min and the theme suggest that you stick to
% the following rules:  

% - Exactly two or three sections (other than the summary).
% - At *most* three subsections per section.
% - Talk about 30s to 2min per frame. So there should be between about
%   15 and 30 frames, all told.

\section{Real Analysis Lecture 10}
\subsection{More on Closed Sets}

\begin{frame}{Adherent and Accumulation Points}
\begin{defn}
A point $\vec x\in \mathbb{R}^n$ is called an \textbf{adherent point} if for all $r > 0$ the ball $B(\vec x; r)$ contains at least one element of $A$.\\
\pause
It is called an \textbf{accumulation point} if for all $r > 0$ the ball $B(\vec x; r)$ contains at least one element of $A$ {\color{red} different from $\vec x$}.
\end{defn}
\pause
\textbf{Examples:}
\begin{itemize}
\pause
\item every point in $A$ is adherent
\pause
\item accumulation points are adherent points
\pause
\item $-1$ is an accumulation point of $(-1,1)$.
\pause
\item suprema and infima are accumulation points!
\pause
\item $0$ is an accumulation point of $\{1/1,1/2,1/3,\dots\}$
\end{itemize}
\end{frame}

\begin{frame}{Characterizing accumulation points}
\begin{thm}[Apostol Theorem 3.17]
A point $\vec x$ is an accumulation point of $A$ if and only if for all $r > 0$, the ball $B(\vec x; r)$ contains infinitely many points of $A$.
\end{thm}
\pause
\begin{proof}
\pause
Obviously, if for all $r > 0$, the ball $B(\vec x; r)$ contains infinitely many points of $A$, then it contains at least one point of $A$ different from $\vec x$, so it's an accumulation point.\\
\pause
Suppose instead that $\vec x$ is an accumulation point and let $r > 0$.\\
\pause
It suffices to show that the set $C = (B(\vec x; r)\backslash\{\vec x\})\cap A$ is infinite.\\
\end{proof}
\end{frame}

\begin{frame}{Characterizing accumulation points}
\begin{thm}[Apostol Theorem 3.17]
A point $\vec x$ is an accumulation point of $A$ if for all $r > 0$, the ball $B(\vec x; r)$ contains infinitely many points of $A$.
\end{thm}
\pause
\begin{proof}
\pause
Suppose that $C$ is finite.\\
\pause
Then the set $\{|\vec y - \vec x|: \vec y\in C\}$ is also finite.\\
\pause
It is also nonempty and has only positive values.\ \pause This means that it has a minimum value $s > 0$.\\
\pause
Since $\vec x$ is an accumulation point, $B(\vec x; s)\cap A$ has an element $\vec y$ different from $\vec x$.\\
\pause
However, $\vec y\in C$ and $|\vec y - \vec x| < s$, contradicting the minimality of $s$.\\
\pause
We conclude that $C$ is infinite.
\end{proof}
\end{frame}


\begin{frame}{Closure of a set}
\begin{defn}
The \textbf{closure} of a set $A$ is the set $\overline{A}$ of all adherent points of $A$
\end{defn}
\pause
\begin{thm}[Apostol Theorem 3.18, 3.20, 3.22]
A set is closed if and only if it contains all of its adherent points (ie. $A= \overline{A}$), or equivalently if and only if $A$ contains all of its accumulation points. 
\end{thm}
\end{frame}

\begin{frame}{Closure of a set}
\begin{proof}
Suppose that $A$ is closed.\\
\pause
Then $\mathbb{R}^n\backslash A$ is open.\\
\pause
If $\vec x\in \mathbb{R}^n\backslash A$, then there exist $r > 0$ such that $B(\vec x; r)\subseteq \mathbb{R}^n\backslash A$.\\
\pause
This means that $B(\vec x; r)\cap A = \varnothing$.\\
\pause
Thus $\vec x$ is not an adherent point of $A$.\\
\pause
Thus every adherent point of $A$ is an element of $A$.\\
\pause
Conversely, suppse that $A$ contains all of its accumulation points.\\
\pause
If $\vec x\in \mathbb R^n\backslash A$, then $\vec x$ is not an accumulation point.\\
\pause
Therefore there exists $r > 0$ with $B(\vec x; r)\cap A = \varnothing$.\\
\pause
This implies that $B(\vec x; r)\subseteq \mathbb R^n\backslash A$.\\
\pause
Thus $\vec x$ is an interior point of $\mathbb R^n\backslash A$.\\
\pause
Since $\vec x$ was an arbitrary element of $\mathbb R^n\backslash A$, this proves $\mathbb R^n\backslash A$ is open.\\
\pause
Therefore $A$ is closed.
\end{proof}
\end{frame}

\begin{frame}{Closures are closed}
\begin{thm}
If $A\subseteq \mathbb{R}^n$ is any set, then $\overline{A}$ is closed.
\end{thm}
\begin{proof}
Exercise!
\end{proof}
\end{frame}

\end{document}


