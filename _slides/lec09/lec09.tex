% !TEX TS-program = pdflatex
% !TEX encoding = UTF-8 Unicode

% This file is a template using the "beamer" package to create slides for a talk or presentation
% - Giving a talk on some subject.
% - The talk is between 15min and 45min long.
% - Style is ornate.

% MODIFIED by Jonathan Kew, 2008-07-06
% The header comments and encoding in this file were modified for inclusion with TeXworks.
% The content is otherwise unchanged from the original distributed with the beamer package.

\documentclass{beamer}
\setbeamercovered{invisible}



% Copyright 2004 by Till Tantau <tantau@users.sourceforge.net>.
%
% In principle, this file can be redistributed and/or modified under
% the terms of the GNU Public License, version 2.
%
% However, this file is supposed to be a template to be modified
% for your own needs. For this reason, if you use this file as a
% template and not specifically distribute it as part of a another
% package/program, I grant the extra permission to freely copy and
% modify this file as you see fit and even to delete this copyright
% notice. 


\mode<presentation>
{
  \usetheme{Warsaw}
  % or ...

  % or whatever (possibly just delete it)
}


\usepackage[english]{babel}
% or whatever

\usepackage[utf8]{inputenc}
% or whatever

\usepackage{times}
\usepackage[T1]{fontenc}
% Or whatever. Note that the encoding and the font should match. If T1
% does not look nice, try deleting the line with the fontenc.

%%% MATH RELATED


%%% AMS math stuff
\usepackage{amsmath}
\usepackage{amssymb}
\usepackage{amsthm}
\usepackage{mathrsfs}
\usepackage{enumerate}

%%% Theorem environments
\newtheorem{thm}{Theorem}[subsection]
\newtheorem{prop}[thm]{Proposition}
\newtheorem{cor}{Corollary}[thm]
\newtheorem{por}[thm]{Porism}
\newtheorem{lem}[thm]{Lemma}
\theoremstyle{definition}
\newtheorem{prob}[thm]{Problem}
\newtheorem{soln}{Solution}
\newtheorem{defn}[thm]{Definition}
\newtheorem{ex}[thm]{Example}
\newtheorem{quest}[thm]{Question}
\newtheorem{remk}[thm]{Remark}

%%% Typesetting shortcuts
\newcommand{\tn}[1]{\textnormal{#1}}
\newcommand{\ol}[1]{\overline{#1}}
\newcommand{\wt}[1]{\widetilde{#1}}
\newcommand{\wh}[1]{\widehat{#1}}
\newcommand{\vocab}[1]{\textbf{#1}\index{#1}}

%%% Math shortcuts
\newcommand{\bbr}{\mathbb R}
\newcommand{\bbz}{\mathbb Z}
\newcommand{\bbq}{\mathbb Q}
\newcommand{\bbn}{\mathbb N}
\newcommand{\bbf}{\mathbb F}
\newcommand{\bbc}{\mathbb C}
\newcommand{\bbd}{\mathbb D}
\newcommand{\bba}{\mathbb A}
\newcommand{\bbp}{\mathbb P}
\newcommand{\bbg}{\mathbb G}
\newcommand{\bbv}{\mathbb V}
\newcommand{\dih}[1]{\mathcal D_{#1}}
\newcommand{\sym}[1]{\mathcal S_{#1}}
\newcommand{\vspan}{\tn{span}}
\newcommand{\trace}{\tn{trace}}
\newcommand{\diff}{\backslash}
\newcommand{\stab}{\tn{stab}}
\newcommand{\conv}{\tn{conv}}
\newcommand{\img}{\tn{img}}
\newcommand{\coker}{\tn{coker}}
\newcommand{\id}{\tn{id}}
\newcommand{\Hom}{\tn{Hom}}
\newcommand{\End}{\tn{End}}
\newcommand{\Aut}{\tn{Aut}}
\newcommand{\aut}{\tn{Aut}}
\newcommand{\ann}{\tn{Ann}}
\newcommand{\GL}{\tn{GL}}
\newcommand{\Gr}{\tn{Gr}}
\newcommand{\lord}{\preccurlyeq}
\newcommand{\rord}{\succcurlyeq}
\newcommand{\tr}{\textnormal{Tr}}
\newcommand{\Tr}{\textnormal{Tr}}
\newcommand{\bbl}{\mathbb{L}}
\newcommand{\C}{\mathscr{C}}
\newcommand{\X}{\mathscr{X}}
\renewcommand{\S}{\mathscr{S}}
\newcommand{\M}{\mathscr{M}}
\renewcommand{\L}{\mathcal{L}}


%%% Algebraic Geometry
\newcommand{\height}{\textnormal{ht}}
\newcommand{\A}{\mathbb{A}}
\newcommand{\p}{\mathfrak{p}}
\newcommand{\sheaf}[1]{\mathcal{#1}}
\newcommand{\spec}{\textnormal{Spec}}
\newcommand{\proj}{\textnormal{Proj}}
\newcommand{\Aff}{\textnormal{Aff}}
\newcommand{\skel}{\textnormal{skel}}
\newcommand{\supp}{\textnormal{supp}}
\newcommand{\orb}{\textnormal{orb}}
\newcommand{\Proj}{\textnormal{Proj}}
\newcommand{\Pic}{\textnormal{Pic}}
\newcommand{\Rees}{\textnormal{Rees}}
\newcommand{\shom}{\mathcal{H}om}

\renewcommand*\arraystretch{1.3}

%%% paper specific definitions
\newcommand{\weyl}{\Omega}
\newcommand{\weyll}{{\widehat{\Omega}}}
\newcommand{\weylll}{{\widetilde{\Omega}}}
\newcommand{\mweyl}{{M_N(\Omega)}}
\newcommand{\mweyll}{{M_N(\widehat{\Omega})}}
\newcommand{\mweylll}{{M_N(\widetilde{\Omega})}}
\newcommand{\seq}{\text{Seq}}
\newcommand{\tail}{\text{Tail}}
\newcommand{\Ad}{\textnormal{Ad}}
\newcommand{\sech}{\textnormal{sech}}
\newcommand{\colim}{\varinjlim}
\newcommand{\limit}{\varprojlim}
\newcommand{\Bis}{\textnormal{Bis}}
\newcommand{\m}{\mathfrak{m}}
\newcommand{\mxx}[4]{\left(\begin{array}{cc} #1 & #2\\ #3 & #4 \end{array}\right)}
\newcommand{\diag}{\text{diag}}
\newcommand{\qdet}{\textnormal{qdet}}
\newcommand{\mdet}{\textnormal{mdet}}
\newcommand{\mtau}{\mathcal{T}}
\newcommand{\cof}{\textnormal{cof}}
\newcommand{\minor}{\textnormal{minor}}
\newcommand{\holo}{Holo}
\newcommand{\ord}{\textnormal{order}}
\newcommand{\mult}{\mathfrak M}






\title{MATH 350-2 Advanced Calculus} 
\subtitle
{} % (optional)

\author[W.R. Casper] % (optional, use only with lots of authors)
{W.R. Casper}
% - Use the \inst{?} command only if the authors have different
%   affiliation.

\institute[California State University Fullerton] % (optional, but mostly needed)
{
  Department of Mathematics\\
  California State University Fullerton}
% - Use the \inst command only if there are several affiliations.
% - Keep it simple, no one is interested in your street address.

\subject{Talks}
% This is only inserted into the PDF information catalog. Can be left
% out. 



% If you have a file called "university-logo-filename.xxx", where xxx
% is a graphic format that can be processed by latex or pdflatex,
% resp., then you can add a logo as follows:

% \pgfdeclareimage[height=0.5cm]{university-logo}{university-logo-filename}
% \logo{\pgfuseimage{university-logo}}



% Delete this, if you do not want the table of contents to pop up at
% the beginning of each subsection:
\AtBeginSubsection[]
{
  \begin{frame}<beamer>{Outline}
    \tableofcontents[currentsection,currentsubsection]
  \end{frame}
}


% If you wish to uncover everything in a step-wise fashion, uncomment
% the following command: 

%\beamerdefaultoverlayspecification{<+->}


\begin{document}

\begin{frame}
  \titlepage
\end{frame}

\begin{frame}{Outline}
  \tableofcontents
  % You might wish to add the option [pausesections]
\end{frame}

% Since this a solution template for a generic talk, very little can
% be said about how it should be structured. However, the talk length
% of between 15min and 45min and the theme suggest that you stick to
% the following rules:  

% - Exactly two or three sections (other than the summary).
% - At *most* three subsections per section.
% - Talk about 30s to 2min per frame. So there should be between about
%   15 and 30 frames, all told.

\section{Real Analysis Lecture 9}
\subsection{More on Open Sets}

\begin{frame}{Finite intersections of open sets are open}
\begin{thm}[Open Intersection Theorem]
Suppose that $U_1,U_2,\dots,U_n$ are open sets.
Then $\bigcap_{i=1}^n U_i$ is open.
\end{thm}
\pause
\begin{proof}
\pause
Suppose that $\vec x\in \bigcap_{i=1}^n U_i$.\\
\pause
Then for all $i$, $\vec x\in U_i$.\\
\pause
Since $U_i$ is open, this means $\vec x$ is an interior point of $U_i$.\\
\pause
Therefore there exists $r_i>0$ such that $B(\vec x; r_i)\subseteq U_i$.\\
\pause
Take $r = \min\{r_i: 1\leq i \leq n\}$.\\
\pause
This means $B(\vec x; r)\subseteq B(\vec x; r_i)\subseteq U_i$ for all $i$.\\
\pause
Therefore $B(\vec x; r)\subseteq \bigcap_{i=1}^n U_i$.\\
\pause
Thus $\vec x$ is an interior point of $\bigcap_{i=1}^n U_i$.\\
\pause
Since $\vec x$ is arbitrary, this proves that $\bigcap_{i=1}^n U_i$ is open.
\end{proof}
\end{frame}

\begin{frame}{Challenge}
\begin{prob}
Show that the infinite intersection
$$\bigcap_{n=1}^\infty (-\frac{1}{n},\frac{1}{n})$$
is not open.
\end{prob}
\end{frame}

\begin{frame}{Component intervals}
Let $U\subseteq \mathbb{R}$ be open.
\pause
\begin{defn}
A \textbf{component interval} of $U$ is an interval $I$ with $I\subseteq U$ and with the property that if $J$ is an interval and $I\subsetneq J$, then $J\nsubseteq U$.
\end{defn}
\begin{itemize}
\pause
\item $(0,1)$ is a component interval of $\mathbb{R}\diff \{0,1,2,3\}$
\pause
\item $(-\infty,0)$ is also a component interval of $\mathbb{R}\diff \{0,1,2,3\}$
\pause
\item we will show all open sets of $\mathbb{R}$ are made of component intervals!
\end{itemize}
\end{frame}

\begin{frame}{Component intervals}
\begin{lemma}
If $I_1$ and $I_2$ are two component intervals of an open subset $U\subseteq\mathbb{R}$, then $I_1=I_2$ or $I_1\cap I_2 = \varnothing$.
\end{lemma}
\pause
\begin{proof}
\pause
Suppose that $I_1\cap I_2 \neq \varnothing$.\\
\pause
Then $J := I_1\cup I_2$ is an interval.\\
\pause
Also $I_1\subseteq U$ and $I_2\subseteq U$, so $J\subseteq U$.\\
\pause
Since $I_i$ is a component interval and $I_i\subseteq J\subseteq U$, we have $I_i = J$.\\
\pause
In particular $I_1 = J = I_2$.
\end{proof}
\end{frame}

\begin{frame}{Component intervals}
\begin{thm}[Apostol 3.10]
If $U\subseteq\mathbb{R}$ is open and $x\in U$, then there is a unique component interval of $U$ containing $x$.
\end{thm}
\pause
\begin{proof}
Uniqueness follows from previous Lemma, so we only need existence.\\
\pause
Suppose that $x\in U$ and consider
$$A = \{a: (a,x)\subseteq U\},\quad\text{and}\quad B = \{b: (x,b)\subseteq U\}$$
\pause
If $A$ is not bounded below, let $a=-\infty$.  Otherwise, let $a = \inf (A)$.\\
\pause
If $B$ is not bounded above, let $b=\infty$.  Otherwise, let $b = \sup (B)$.\\
\pause
Claim: $(a,b)$ is a component interval of $U$ containing $x$.
\end{proof}
\end{frame}

\begin{frame}{Challenge}
\begin{prob}
Prove that $(a,b)$ is a component interval of $U$.
\end{prob}
\end{frame}

\begin{frame}{Open subsets of $\mathbb{R}$}
\begin{thm}[Representation Theorem for Open Sets in $\mathbb{R}$]
If $U\subseteq\mathbb{R}$ is open, then $U$ is the union of a countable family of disjoint open intervals.
\end{thm}
\pause
\begin{proof}
From the previous theorem, each $x\in U$ is contained in a unique component interval $I_x$ of $U$.\\
\pause
Therefore
$$U = \bigcup_{x\in U} I_x$$
is a union of open intervals!\\
\pause
{\color{red}Uh oh ... this isn't a \textbf{countable} union.}
\end{proof}
\end{frame}

\begin{frame}{Open subsets of $\mathbb{R}$}
\begin{thm}[Representation Theorem for Open Sets in $\mathbb{R}$]
If $U\subseteq\mathbb{R}$ is open, then $U$ is the union of a countable family of disjoint open intervals.
\end{thm}
\begin{proof}
Instead, we consider rational points $U\cap\mathbb{Q}$.\\
\pause
If $x\in U$, then $x\in I_x\subseteq U$.\\
\pause
Any interval contains a rational number $r\in I_x$.\\
\pause
Since different component intervals must be disjoint, $I_x=I_r$.\\
\pause
Thus $x\in \bigcup_{r\in U\cap\mathbb{Q}} I_r\subseteq U$.\\
\pause
Since $x$ was arbitrary, 
$$U= \bigcup_{r\in U\cap\mathbb{Q}} I_r$$
\end{proof}
\end{frame}

\subsection{Closed Sets}

\begin{frame}{Closed sets}
\begin{defn}
A set $A\subseteq\mathbb{R}^n$ is called \vocab{closed} if it is the complement of an open set, or equivalently if its complement $\mathbb{R}^n\backslash A$ is open.
\end{defn}
\pause
\textbf{Examples:}
\begin{itemize}
\pause
\item singleton sets!
\pause
\item products of closed intervals
\end{itemize}
\end{frame}

\begin{frame}{Challenge!}
\begin{prob}
Prove that a singleton set
$$A = \{\vec a\}$$
in $\mathbb{R}^n$ is closed.
\end{prob}
\pause
\begin{soln}
We must show $U = \mathbb{R}^n\backslash\{\vec a\}$ is open.\\\pause
Suppose $\vec x\in U$.\ \pause
We must show $\vec x$ is an interior point of $U$.\\\pause
Take $r = |\vec x-\vec a|$.\ \pause Then $B(\vec x; r)$ does not contain $\vec a$.\\\pause
Therefore $B(\vec x; r)\subseteq U$, so that $\vec x$ is an interior point.\\\pause
Since $\vec x$ was arbitrary, this proves $U$ is open.
\end{soln}
\end{frame}

\begin{frame}{Challenge!}
\begin{prob}
Prove that the \textbf{closed square}
$$[a,b]\times [c,d] = \{(x,y)\in\mathbb{R}^2: a \leq x \leq b,\ c\leq y\leq d\}$$
is a closed set.
\end{prob}
\pause
\begin{soln}
$\mathbb{R}^2\backslash ([a,b]\times [c,d])$ is the same as
$$((-\infty,a)\times \mathbb{R})\cup ((b,\infty)\times \mathbb{R})\cup(\mathbb{R}\times (-\infty,c))\cup (\mathbb{R}\times (d,\infty))$$
\pause
Union of open sets is open!
\end{soln}
\end{frame}


\begin{frame}{Adherent and Accumulation Points}
\begin{defn}
A point $\vec x\in A$ is called an \textbf{adherent point} if for all $r > 0$ the ball $B(\vec x; r)$ contains at least one element of $A$.\\
\pause
It is called an \textbf{accumulation point} if for all $r > 0$ the ball $B(\vec x; r)$ contains at least one element of $A$ {\color{red} different from $\vec x$}.
\end{defn}
\pause
\textbf{Examples:}
\begin{itemize}
\pause
\item every point in $A$ is adherent
\pause
\item accumulation points are adherent points
\pause
\item $-1$ is an accumulation point of $(-1,1)$.
\pause
\item suprema and infima are accumulation points!
\pause
\item $0$ is an accumulation point of $\{1/1,1/2,1/3,\dots\}$
\end{itemize}
\end{frame}



\end{document}


