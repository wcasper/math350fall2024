% !TEX TS-program = pdflatex
% !TEX encoding = UTF-8 Unicode

% This file is a template using the "beamer" package to create slides for a talk or presentation
% - Giving a talk on some subject.
% - The talk is between 15min and 45min long.
% - Style is ornate.

% MODIFIED by Jonathan Kew, 2008-07-06
% The header comments and encoding in this file were modified for inclusion with TeXworks.
% The content is otherwise unchanged from the original distributed with the beamer package.

\documentclass{beamer}
\setbeamercovered{invisible}



% Copyright 2004 by Till Tantau <tantau@users.sourceforge.net>.
%
% In principle, this file can be redistributed and/or modified under
% the terms of the GNU Public License, version 2.
%
% However, this file is supposed to be a template to be modified
% for your own needs. For this reason, if you use this file as a
% template and not specifically distribute it as part of a another
% package/program, I grant the extra permission to freely copy and
% modify this file as you see fit and even to delete this copyright
% notice. 


\mode<presentation>
{
  \usetheme{Warsaw}
  % or ...

  % or whatever (possibly just delete it)
}


\usepackage[english]{babel}
% or whatever

\usepackage[utf8]{inputenc}
% or whatever

\usepackage{times}
\usepackage[T1]{fontenc}
% Or whatever. Note that the encoding and the font should match. If T1
% does not look nice, try deleting the line with the fontenc.

%%% MATH RELATED


%%% AMS math stuff
\usepackage{amsmath}
\usepackage{amssymb}
\usepackage{amsthm}
\usepackage{mathrsfs}
\usepackage{enumerate}

%%% Theorem environments
\newtheorem{thm}{Theorem}[subsection]
\newtheorem{prop}[thm]{Proposition}
\newtheorem{cor}{Corollary}[thm]
\newtheorem{por}[thm]{Porism}
\newtheorem{lem}[thm]{Lemma}
\theoremstyle{definition}
\newtheorem{prob}[thm]{Problem}
\newtheorem{soln}{Solution}
\newtheorem{defn}[thm]{Definition}
\newtheorem{ex}[thm]{Example}
\newtheorem{quest}[thm]{Question}
\newtheorem{remk}[thm]{Remark}

%%% Typesetting shortcuts
\newcommand{\tn}[1]{\textnormal{#1}}
\newcommand{\ol}[1]{\overline{#1}}
\newcommand{\wt}[1]{\widetilde{#1}}
\newcommand{\wh}[1]{\widehat{#1}}
\newcommand{\vocab}[1]{\textbf{#1}\index{#1}}

%%% Math shortcuts
\newcommand{\bbr}{\mathbb R}
\newcommand{\bbz}{\mathbb Z}
\newcommand{\bbq}{\mathbb Q}
\newcommand{\bbn}{\mathbb N}
\newcommand{\bbf}{\mathbb F}
\newcommand{\bbc}{\mathbb C}
\newcommand{\bbd}{\mathbb D}
\newcommand{\bba}{\mathbb A}
\newcommand{\bbp}{\mathbb P}
\newcommand{\bbg}{\mathbb G}
\newcommand{\bbv}{\mathbb V}
\newcommand{\dih}[1]{\mathcal D_{#1}}
\newcommand{\sym}[1]{\mathcal S_{#1}}
\newcommand{\vspan}{\tn{span}}
\newcommand{\trace}{\tn{trace}}
\newcommand{\diff}{\backslash}
\newcommand{\stab}{\tn{stab}}
\newcommand{\conv}{\tn{conv}}
\newcommand{\img}{\tn{img}}
\newcommand{\coker}{\tn{coker}}
\newcommand{\id}{\tn{id}}
\newcommand{\Hom}{\tn{Hom}}
\newcommand{\End}{\tn{End}}
\newcommand{\Aut}{\tn{Aut}}
\newcommand{\aut}{\tn{Aut}}
\newcommand{\ann}{\tn{Ann}}
\newcommand{\GL}{\tn{GL}}
\newcommand{\Gr}{\tn{Gr}}
\newcommand{\lord}{\preccurlyeq}
\newcommand{\rord}{\succcurlyeq}
\newcommand{\tr}{\textnormal{Tr}}
\newcommand{\Tr}{\textnormal{Tr}}
\newcommand{\bbl}{\mathbb{L}}
\newcommand{\C}{\mathscr{C}}
\newcommand{\X}{\mathscr{X}}
\renewcommand{\S}{\mathscr{S}}
\newcommand{\M}{\mathscr{M}}
\renewcommand{\L}{\mathcal{L}}


%%% Algebraic Geometry
\newcommand{\height}{\textnormal{ht}}
\newcommand{\A}{\mathbb{A}}
\newcommand{\p}{\mathfrak{p}}
\newcommand{\sheaf}[1]{\mathcal{#1}}
\newcommand{\spec}{\textnormal{Spec}}
\newcommand{\proj}{\textnormal{Proj}}
\newcommand{\Aff}{\textnormal{Aff}}
\newcommand{\skel}{\textnormal{skel}}
\newcommand{\supp}{\textnormal{supp}}
\newcommand{\orb}{\textnormal{orb}}
\newcommand{\Proj}{\textnormal{Proj}}
\newcommand{\Pic}{\textnormal{Pic}}
\newcommand{\Rees}{\textnormal{Rees}}
\newcommand{\shom}{\mathcal{H}om}

\renewcommand*\arraystretch{1.3}

%%% paper specific definitions
\newcommand{\weyl}{\Omega}
\newcommand{\weyll}{{\widehat{\Omega}}}
\newcommand{\weylll}{{\widetilde{\Omega}}}
\newcommand{\mweyl}{{M_N(\Omega)}}
\newcommand{\mweyll}{{M_N(\widehat{\Omega})}}
\newcommand{\mweylll}{{M_N(\widetilde{\Omega})}}
\newcommand{\seq}{\text{Seq}}
\newcommand{\tail}{\text{Tail}}
\newcommand{\Ad}{\textnormal{Ad}}
\newcommand{\sech}{\textnormal{sech}}
\newcommand{\colim}{\varinjlim}
\newcommand{\limit}{\varprojlim}
\newcommand{\Bis}{\textnormal{Bis}}
\newcommand{\m}{\mathfrak{m}}
\newcommand{\mxx}[4]{\left(\begin{array}{cc} #1 & #2\\ #3 & #4 \end{array}\right)}
\newcommand{\diag}{\text{diag}}
\newcommand{\qdet}{\textnormal{qdet}}
\newcommand{\mdet}{\textnormal{mdet}}
\newcommand{\mtau}{\mathcal{T}}
\newcommand{\cof}{\textnormal{cof}}
\newcommand{\minor}{\textnormal{minor}}
\newcommand{\holo}{Holo}
\newcommand{\ord}{\textnormal{order}}
\newcommand{\mult}{\mathfrak M}






\title{MATH 350-2 Advanced Calculus} 
\subtitle
{} % (optional)

\author[W.R. Casper] % (optional, use only with lots of authors)
{W.R. Casper}
% - Use the \inst{?} command only if the authors have different
%   affiliation.

\institute[California State University Fullerton] % (optional, but mostly needed)
{
  Department of Mathematics\\
  California State University Fullerton}
% - Use the \inst command only if there are several affiliations.
% - Keep it simple, no one is interested in your street address.

\subject{Talks}
% This is only inserted into the PDF information catalog. Can be left
% out. 



% If you have a file called "university-logo-filename.xxx", where xxx
% is a graphic format that can be processed by latex or pdflatex,
% resp., then you can add a logo as follows:

% \pgfdeclareimage[height=0.5cm]{university-logo}{university-logo-filename}
% \logo{\pgfuseimage{university-logo}}



% Delete this, if you do not want the table of contents to pop up at
% the beginning of each subsection:
\AtBeginSubsection[]
{
  \begin{frame}<beamer>{Outline}
    \tableofcontents[currentsection,currentsubsection]
  \end{frame}
}


% If you wish to uncover everything in a step-wise fashion, uncomment
% the following command: 

%\beamerdefaultoverlayspecification{<+->}


\begin{document}

\begin{frame}
  \titlepage
\end{frame}

\begin{frame}{Outline}
  \tableofcontents
  % You might wish to add the option [pausesections]
\end{frame}

% Since this a solution template for a generic talk, very little can
% be said about how it should be structured. However, the talk length
% of between 15min and 45min and the theme suggest that you stick to
% the following rules:  

% - Exactly two or three sections (other than the summary).
% - At *most* three subsections per section.
% - Talk about 30s to 2min per frame. So there should be between about
%   15 and 30 frames, all told.

\section{Real Analysis Lecture 7}

\subsection{More with cardinality}

\begin{frame}{Ultimate Cantor diagonalization}
The \textbf{power set}  $\mathcal{P}(A)$ of a set $A$ is
$$\mathcal P(A) = \{S: S\subseteq A\}.$$
\pause
\begin{thm}
The cardinality of $\mathcal P(A)$ is strictly larger than $A$.
\end{thm}
\pause
\begin{proof}
\pause
Suppose $f: A\rightarrow \mathcal P(A)$ is surjective.\\
\pause
Consider the set
$$S = \{a\in A: a\notin f(a)\}.$$
\end{proof}
\end{frame}

\begin{frame}{Challenge}
\pause
Since $f$ is surjectve, $S = f(x)$ for some $x\in A$.
\pause
\begin{prob}
Consider the statement $x\in S$.  What can you conclude?
\end{prob}
\end{frame}

\begin{frame}{Challenge}
\begin{prob}
Is there a set with cardinality larger than $\mathbb{R}$?
\end{prob}
\end{frame}

\begin{frame}{Challenge}
\begin{prob}
Show that the cardinality of the line segment $(0,1)$ and the square $(0,1)\times (0,1)$ is the same.
\end{prob}
\end{frame}

\subsection{Set algebra}

\begin{frame}{Unions of sets}
Unions:
\pause
$$A\cup B = \{x: x\in A\ \ \text{or}\ \ x\in B\},$$
\pause
$$A\cup B\cup C = \{x: x\in A\ \ \text{or}\ \ x\in B\ \ \text{or}\ \ x\in C\},$$
\pause
$$A_1\cup A_2\cup \dots \cup A_n = \{x: x\in A_i\ \ \text{for some $1\leq i\leq n$}\}$$
\pause
{\color{red}NOTATION:}
$$\bigcup_{i=1}^n A_i = A_1\cup A_2\cup \dots \cup A_n.$$
\end{frame}

\begin{frame}{Intersections of sets}
Intersections:
\pause
$$A\cap B = \{x: x\in A\ \ \text{and}\ \ x\in B\},$$
\pause
$$A\cap B\cap C = \{x: x\in A\ \ \text{and}\ \ x\in B\ \ \text{and}\ \ x\in C\},$$
\pause
$$A_1\cap A_2\cap \dots \cap A_n = \{x: x\in A_i\ \ \text{for some $1\leq i\leq n$}\}$$
\pause
{\color{red}NOTATION:}
$$\bigcap_{i=1}^n A_i = A_1\cap A_2\cap \dots \cap A_n.$$
\end{frame}

\begin{frame}{Complements of sets}
\pause
The \vocab{complement} of $A$ relative to $B$ is
$$B-A = \{b\in B: b\notin A\}.$$
\pause
\begin{thm}{De Morgan's Laws}
\pause
$$B-\bigcup_{i=1}^n A_i = \bigcap_{i=1}^n (B-A_i)$$
\pause
$$B-\bigcap_{i=1}^n A_i = \bigcup_{i=1}^n (B-A_i)$$
\end{thm}
\end{frame}


\begin{frame}{Families of sets}
A \textbf{family of sets} is a collection
$$\{A_i; i\in I\}$$
\begin{itemize}
\pause
\item $I$ any set, called the \textbf{index set}
\pause
\item $A_i$ is a set for all $i$
\end{itemize}
\pause
$$\bigcup_{i\in I} A_i = \{x: x\in A_i,\ \text{for some $i\in I$}\}$$
\pause
$$\bigcap_{i\in I} A_i = \{x: x\in A_i,\ \text{for all $i\in I$}\}$$
\end{frame}

\begin{frame}{Challenge}
\begin{itemize}
\item index set $I = (1,2)$
\item family of sets $\{A_i: i\in I\}$
\item $A_i = [0,i]$
\end{itemize}
\begin{prob}
Determine $\bigcup_{i\in I} A_i$.
\end{prob}
\end{frame}

\begin{frame}{Challenge}
\begin{itemize}
\item index set $I = \mathbb{Z}_+$
\item family of sets $\{A_i: i\in I\}$
\item $A_i = [0,i)$
\end{itemize}
\begin{prob}
Determine $\bigcap_{i\in I} A_i$.
\end{prob}
\end{frame}

\begin{frame}{Countable unions}
A \textbf{countable family of sets} is a family of sets where the index set is countable.
\pause
\begin{thm}
The union of a countable family of countable sets is countable.
\end{thm}
\pause
\begin{proof}
Let $\{A_i: i\in I\}$ be a countable family of countable sets.\\
\pause
$I$ is countable.\ \pause There is a surjection $\mathbb{Z}_+\rightarrow I$\\
\pause
In other words, we can enumerate $I = \{i_1,i_2,i_3,\dots\}$.\\
\pause
$A_i$ is countable for all $i\in I$.\\
\pause
So again $A_i = \{a_{i1},a_{i2},a_{i3},\dots\}$.\\
\pause
Define $f: \mathbb Z_+\times\mathbb Z_+\rightarrow \bigcup_{i\in I} A_i$\\
\pause
$f(j,k) = a_{i_j,k}$.\ \pause Surjection!
\end{proof}
\end{frame}

\begin{frame}{Challenge!}
A real number is called \textbf{algebraic} if it is a root of a polynomial with integer coefficients.
\pause
\begin{prob}
Prove that the set of all algebraic numbers is countable.
\end{prob}
\pause
Hint: use the previous theorem!
\end{frame}

\subsection{Open Balls and Open Sets}

\begin{frame}{Euclidean space}
\pause
$n$-dimensional \textbf{euclidean space} is
\pause
$$\mathbb{R}^n = \{(a_1,a_2,\dots,a_n): a_1,\dots, a_n\in\mathbb{R}\}.$$
\pause
Definitions:\\
\pause
Given $\vec x = (x_1,\dots, x_n)$ and $\vec y = (y_1,\dots, y_n)$
\begin{enumerate}[(a)]
\pause
\item $\vec x + \vec y = (x_1+y_1,\dots,x_n+y_n)$
\pause
\item $a\vec x = (ax_1,\dots, ax_n)$
\pause
\item $\vec x - \vec y = \vec x + (-1)\vec y = (x_1-y_1,\dots,x_n-y_n)$
\pause
\item $\vec 0 = (0,\dots, 0)$
\pause
\item $\vec x\cdot\vec y = \sum_{i=1}^n x_iy_y$
\pause
\item $|\vec x| = (\vec x\cdot\vec x)^{1/2} = \left(\sum_{i=1}^nx_i^2\right)^{1/2}$
\end{enumerate}
\end{frame}

\begin{frame}{Metric properties}
\pause
The norm $|\vec x|$ is an example of a \vocab{metric}.\\
\pause
It satisfies several important properties:
\pause
\begin{thm}
\begin{enumerate}[(a)]
\pause
\item (positivity) $|\vec x| \geq 0$ with equality iff $\vec x = \vec 0$
\pause
\item (symmetry) $|\vec x + \vec y| = |\vec y + \vec x|$
\pause
\item (triangle inequality) $|\vec x + \vec y| \leq |\vec x| + | \vec y|$
\end{enumerate}
\end{thm}
\pause
It also satisfies 
\pause
\begin{enumerate}[(a)]
\pause
\item (scaling)  $|c\vec x| = |c|\ |\vec x|$
\pause
\item (Cauchy-Schwartz) $|\vec x\cdot\vec y|\leq |\vec x|\ |\vec y|$
\end{enumerate}
\end{frame}

\end{document}


