% !TEX TS-program = pdflatex
% !TEX encoding = UTF-8 Unicode

% This file is a template using the "beamer" package to create slides for a talk or presentation
% - Giving a talk on some subject.
% - The talk is between 15min and 45min long.
% - Style is ornate.

% MODIFIED by Jonathan Kew, 2008-07-06
% The header comments and encoding in this file were modified for inclusion with TeXworks.
% The content is otherwise unchanged from the original distributed with the beamer package.

\documentclass{beamer}
\setbeamercovered{invisible}



% Copyright 2004 by Till Tantau <tantau@users.sourceforge.net>.
%
% In principle, this file can be redistributed and/or modified under
% the terms of the GNU Public License, version 2.
%
% However, this file is supposed to be a template to be modified
% for your own needs. For this reason, if you use this file as a
% template and not specifically distribute it as part of a another
% package/program, I grant the extra permission to freely copy and
% modify this file as you see fit and even to delete this copyright
% notice. 


\mode<presentation>
{
  \usetheme{Warsaw}
  % or ...

  % or whatever (possibly just delete it)
}


\usepackage[english]{babel}
% or whatever

\usepackage[utf8]{inputenc}
% or whatever

\usepackage{times}
\usepackage[T1]{fontenc}
% Or whatever. Note that the encoding and the font should match. If T1
% does not look nice, try deleting the line with the fontenc.

%%% MATH RELATED


%%% AMS math stuff
\usepackage{amsmath}
\usepackage{amssymb}
\usepackage{amsthm}
\usepackage{mathrsfs}
\usepackage{enumerate}

%%% Theorem environments
\newtheorem{thm}{Theorem}[subsection]
\newtheorem{prop}[thm]{Proposition}
\newtheorem{cor}{Corollary}[thm]
\newtheorem{por}[thm]{Porism}
\newtheorem{lem}[thm]{Lemma}
\theoremstyle{definition}
\newtheorem{prob}[thm]{Problem}
\newtheorem{soln}{Solution}
\newtheorem{defn}[thm]{Definition}
\newtheorem{ex}[thm]{Example}
\newtheorem{quest}[thm]{Question}
\newtheorem{remk}[thm]{Remark}

%%% Typesetting shortcuts
\newcommand{\tn}[1]{\textnormal{#1}}
\newcommand{\ol}[1]{\overline{#1}}
\newcommand{\wt}[1]{\widetilde{#1}}
\newcommand{\wh}[1]{\widehat{#1}}
\newcommand{\vocab}[1]{\textbf{#1}\index{#1}}

%%% Math shortcuts
\newcommand{\bbr}{\mathbb R}
\newcommand{\bbz}{\mathbb Z}
\newcommand{\bbq}{\mathbb Q}
\newcommand{\bbn}{\mathbb N}
\newcommand{\bbf}{\mathbb F}
\newcommand{\bbc}{\mathbb C}
\newcommand{\bbd}{\mathbb D}
\newcommand{\bba}{\mathbb A}
\newcommand{\bbp}{\mathbb P}
\newcommand{\bbg}{\mathbb G}
\newcommand{\bbv}{\mathbb V}
\newcommand{\dih}[1]{\mathcal D_{#1}}
\newcommand{\sym}[1]{\mathcal S_{#1}}
\newcommand{\vspan}{\tn{span}}
\newcommand{\trace}{\tn{trace}}
\newcommand{\diff}{\backslash}
\newcommand{\stab}{\tn{stab}}
\newcommand{\conv}{\tn{conv}}
\newcommand{\img}{\tn{img}}
\newcommand{\coker}{\tn{coker}}
\newcommand{\id}{\tn{id}}
\newcommand{\Hom}{\tn{Hom}}
\newcommand{\End}{\tn{End}}
\newcommand{\Aut}{\tn{Aut}}
\newcommand{\aut}{\tn{Aut}}
\newcommand{\ann}{\tn{Ann}}
\newcommand{\GL}{\tn{GL}}
\newcommand{\Gr}{\tn{Gr}}
\newcommand{\lord}{\preccurlyeq}
\newcommand{\rord}{\succcurlyeq}
\newcommand{\tr}{\textnormal{Tr}}
\newcommand{\Tr}{\textnormal{Tr}}
\newcommand{\bbl}{\mathbb{L}}
\newcommand{\C}{\mathscr{C}}
\newcommand{\X}{\mathscr{X}}
\renewcommand{\S}{\mathscr{S}}
\newcommand{\M}{\mathscr{M}}
\renewcommand{\L}{\mathcal{L}}


%%% Algebraic Geometry
\newcommand{\height}{\textnormal{ht}}
\newcommand{\A}{\mathbb{A}}
\newcommand{\p}{\mathfrak{p}}
\newcommand{\sheaf}[1]{\mathcal{#1}}
\newcommand{\spec}{\textnormal{Spec}}
\newcommand{\proj}{\textnormal{Proj}}
\newcommand{\Aff}{\textnormal{Aff}}
\newcommand{\skel}{\textnormal{skel}}
\newcommand{\supp}{\textnormal{supp}}
\newcommand{\orb}{\textnormal{orb}}
\newcommand{\Proj}{\textnormal{Proj}}
\newcommand{\Pic}{\textnormal{Pic}}
\newcommand{\Rees}{\textnormal{Rees}}
\newcommand{\shom}{\mathcal{H}om}

\renewcommand*\arraystretch{1.3}

%%% paper specific definitions
\newcommand{\weyl}{\Omega}
\newcommand{\weyll}{{\widehat{\Omega}}}
\newcommand{\weylll}{{\widetilde{\Omega}}}
\newcommand{\mweyl}{{M_N(\Omega)}}
\newcommand{\mweyll}{{M_N(\widehat{\Omega})}}
\newcommand{\mweylll}{{M_N(\widetilde{\Omega})}}
\newcommand{\seq}{\text{Seq}}
\newcommand{\tail}{\text{Tail}}
\newcommand{\Ad}{\textnormal{Ad}}
\newcommand{\sech}{\textnormal{sech}}
\newcommand{\colim}{\varinjlim}
\newcommand{\limit}{\varprojlim}
\newcommand{\Bis}{\textnormal{Bis}}
\newcommand{\m}{\mathfrak{m}}
\newcommand{\mxx}[4]{\left(\begin{array}{cc} #1 & #2\\ #3 & #4 \end{array}\right)}
\newcommand{\diag}{\text{diag}}
\newcommand{\qdet}{\textnormal{qdet}}
\newcommand{\mdet}{\textnormal{mdet}}
\newcommand{\mtau}{\mathcal{T}}
\newcommand{\cof}{\textnormal{cof}}
\newcommand{\minor}{\textnormal{minor}}
\newcommand{\holo}{Holo}
\newcommand{\ord}{\textnormal{order}}
\newcommand{\mult}{\mathfrak M}






\title{MATH 350-2 Advanced Calculus} 
\subtitle
{} % (optional)

\author[W.R. Casper] % (optional, use only with lots of authors)
{W.R. Casper}
% - Use the \inst{?} command only if the authors have different
%   affiliation.

\institute[California State University Fullerton] % (optional, but mostly needed)
{
  Department of Mathematics\\
  California State University Fullerton}
% - Use the \inst command only if there are several affiliations.
% - Keep it simple, no one is interested in your street address.

\subject{Talks}
% This is only inserted into the PDF information catalog. Can be left
% out. 



% If you have a file called "university-logo-filename.xxx", where xxx
% is a graphic format that can be processed by latex or pdflatex,
% resp., then you can add a logo as follows:

% \pgfdeclareimage[height=0.5cm]{university-logo}{university-logo-filename}
% \logo{\pgfuseimage{university-logo}}



% Delete this, if you do not want the table of contents to pop up at
% the beginning of each subsection:
\AtBeginSubsection[]
{
  \begin{frame}<beamer>{Outline}
    \tableofcontents[currentsection,currentsubsection]
  \end{frame}
}


% If you wish to uncover everything in a step-wise fashion, uncomment
% the following command: 

%\beamerdefaultoverlayspecification{<+->}


\begin{document}

\begin{frame}
  \titlepage
\end{frame}

\begin{frame}{Outline}
  \tableofcontents
  % You might wish to add the option [pausesections]
\end{frame}

% Since this a solution template for a generic talk, very little can
% be said about how it should be structured. However, the talk length
% of between 15min and 45min and the theme suggest that you stick to
% the following rules:  

% - Exactly two or three sections (other than the summary).
% - At *most* three subsections per section.
% - Talk about 30s to 2min per frame. So there should be between about
%   15 and 30 frames, all told.

\section{Real Analysis Lecture 6}

\subsection{Functions}

\begin{frame}{Functions}
A \textbf{function} from $A$ to $B$ is a relation $\mathcal R$ from $A$ to $B$ with the property
\pause
$$a\mathcal Rb\ \text{and}\ a\mathcal Rc\quad\text{implies}\quad b=c.$$
\pause
If a relation $\mathcal R$ is a function, we usually use a symbol like $f$.\\
\pause
\begin{align*}
\text{\color{red}NOTATION:}&\quad f: A\rightarrow B\quad\text{means $f$ is a function from $A$ to $B$}\\
\text{\color{red}NOTATION:}&\quad f(a) = b\quad\text{means}\quad (a,b)\in f.
\end{align*}
\pause
The set $$\img(f) = \{f(a): a\in A\}$$ is called the \textbf{range} or \textbf{image} of $f$
\end{frame}

\begin{frame}{Challenge!}
\begin{prob}
Determine all the equivalence relations on $\mathbb{R}$ which are also functions.
\end{prob}
\pause
\begin{soln}
$f=\mathcal R$ must be reflexive, so $x\mathcal Rx$ for all $x$\\
\pause
This means $f(x) = x$ for all $x$.\\
\pause
Thus the only function which is an equivalence relation is the identity function
$$f(x) = x.$$
\end{soln}
\end{frame}

\begin{frame}{Properties of functions}
A function $f$ from $A$ to $B$ is called \textbf{one-to-one} or \textbf{injective} if 
\pause
$$f(x) = f(y) \quad\text{implies}\quad x=y\ \ \text{for all $x,y\in A$}.$$
\pause
It is called \textbf{onto} or \textbf{surjective} if $\img(f) = B$, or equivalently
$$\text{for all $b\in B$ there exists $a\in A$ with $f(a) =b$}.$$
\pause
If it satisfies both properties, it is called \textbf{bijective}.
\end{frame}

\begin{frame}{Inverses}
The \textbf{converse} of a relation $\mathcal R$ from $A$ to $B$ is the relation
\pause
$$\check{\mathcal R} = \{(b,a)\in B\times A: a\mathcal R b\}.$$
\pause
If $f$ is  function, then $\check f$ may or not be a function.\\
\pause
If it is, we call it the \vocab{inverse} of $f$.
\end{frame}

\begin{frame}{Compositions}
The composition of $f: A\rightarrow B$ and $g: B\rightarrow C$ is the function
\pause
$$g\circ f: A\rightarrow C$$
\pause
with domain $A$ and codomain $C$ defined by
\pause
$$(g\circ f)(x)  = g(f(x)).$$
\end{frame}

\begin{frame}{Sequences}
A \textbf{finite sequence} is a function $f: \{1,2,\dots,n\}\rightarrow \mathbb{R}$.\\
\pause
An \textbf{infinite sequence} is a function $f: \mathbb{Z}_+\rightarrow \mathbb{R}$.\\
\pause
$$\text{\color{red}NOTATION:}\quad f_n\quad\text{indicates the value}\ f(n)$$
\pause
$$\text{\color{red}NOTATION:}\quad \{f_n\}\quad\text{is another way of writing the function}\ f$$
\pause
If $k:\mathbb{Z}_+\rightarrow\mathbb{Z}_+$ is a function which is \vocab{strictly increasing}, meaning
$$m < n\quad\Rightarrow\quad k(m) < k(n),$$
\pause
then the composition $f\circ k: \mathbb{Z}_+\rightarrow \mathbb{R}$ forms a sequence called a \vocab{subsequence} of $f$.
\pause
$$\text{\color{red}NOTATION:}\quad \{f_{k(n)}\}\ \text{or}\ \{f_{k_n}\}\quad\text{both really mean}\ f\circ k$$
\end{frame}

\subsection{Cardinality}

\begin{frame}{Cantor's Paradise}
Philosophy: sets are the same size, or \textbf{cardinality}, if there is a bijection between them.
\begin{itemize}
\pause 
\item $|A|$ denotes the cardinality of $A$
\pause 
\item $|A| \leq |B|$ means there is an injection $f: A\rightarrow B$
\pause 
\item $|A| \geq |B|$ means there is a surjection $f: A\rightarrow B$
\pause 
\item $|A| = |B|$ means there is a bijection $f: A\rightarrow B$
\end{itemize}
\pause
\end{frame}

\begin{frame}{Cantor's Paradise}
\begin{thm}[Cantor-Schroeder-Bernstein Theorem]
The following are equivalent
\begin{enumerate}[(i)]
\item $|A|\leq |B|$  and $|B|\leq |A|$
\item $|A|\geq |B|$  and $|B|\geq |A|$
\item $|A|\leq |B|$  and $|A|\geq |A|$
\item $|A|= |B|$
\end{enumerate}
\end{thm}
\end{frame}

\begin{frame}{Challenge}
\begin{prob}
Show that $\mathbb{Z}_+$ and $\mathbb{Z}_+\times\mathbb{Z}_+$ have the same cardinality.
\end{prob}
\pause
Hint: consider $f: \mathbb{Z}_+\times\mathbb{Z}_+\rightarrow\mathbb{Z}_+$, $f(m,n) = 2^m3^n$
\begin{soln}
\pause
$2^a3^b = 2^c3^d$ if and only if $a=c$ and $b=d$ (Fund. Thm. Arith.)\\
\pause
$f(m,n) = 2^m3^n$ is injective\\
\pause
$|\mathbb{Z}_+\times\mathbb{Z}_+|\leq |\mathbb{Z}_+|$\\
\pause
$g: \mathbb{Z}_+\times\mathbb{Z}_+\rightarrow\mathbb{Z}_+$, $g(m,n) = m$ is surjective\\
\pause
$|\mathbb{Z}_+\times\mathbb{Z}_+|\leq |\mathbb{Z}_+|$
\end{soln}
\end{frame}

\begin{frame}{Challenge}
\begin{prob}
Show that $\mathbb{Z}_+$ and $\mathbb{Z}$ have the same cardinality.
\end{prob}
\pause
Hint: consider $f: \mathbb{Z}_+\rightarrow\mathbb{Z}$
$$f(n) = \left\lbrace\begin{array}{cc}
n/2 & \text{$n$ is even}\\
-(n-)/2 & \text{$n$ is odd}\\
\end{array}\right.$$
\end{frame}

\begin{frame}{Finite and infinite}
\pause
Sets with finite cardinality:
$$\{1,2,3\},\ \ \ \{\mathbb{Z},\mathbb{R}\},\ \ \ \{x:\ \text{$x$ is a student at CSUF}\}$$
\pause
Sets with infinite cardinality:
$$\mathbb{Z}_+,\ \ \ \mathbb{Q},\ \ \ \mathbb{R},\ \ \ \mathbb{C}.$$
\pause
Cantor's discovery: there are multiple sizes of infinity!
$$\mathbb{Z}_+,\ \mathbb{Z},\ \mathbb{Z}\times\mathbb{Z},\ \mathbb{Q},\ \text{are all the same cardinality}$$
\pause
$$\mathbb{R}\  \text{has larger cardinality than}\ \mathbb{Z}_+.$$
\end{frame}


\begin{frame}{Cantor diagonalization}
Suppose there were a bijection
$$f: \mathbb{Z}_+\rightarrow (0,1)$$
\pause
\begin{align*}
f(1) &= 0.a_{11}a_{12}a_{13}a_{14}a_{15}\dots\\
f(2) &= 0.a_{21}a_{22}a_{23}a_{24}a_{25}\dots\\
f(3) &= 0.a_{31}a_{32}a_{33}a_{34}a_{35}\dots\\
f(4) &= 0.a_{41}a_{42}a_{43}a_{44}a_{45}\dots\\
f(5) &= 0.a_{51}a_{52}a_{53}a_{54}a_{55}\dots\\
\vdots 
\end{align*}
\end{frame}

\begin{frame}{Cantor diagonalization}
Suppose there were a surjection
$$f: \mathbb{Z}_+\rightarrow (0,1)$$
\pause
\begin{align*}
f(1) &= 0.{\color{red} a_{11}}a_{12}a_{13}a_{14}a_{15}\dots\\
f(2) &= 0.a_{21}{\color{red} a_{22}}a_{23}a_{24}a_{25}\dots\\
f(3) &= 0.a_{31}a_{32}{\color{red} a_{33}}a_{34}a_{35}\dots\\
f(4) &= 0.a_{41}a_{42}a_{43}{\color{red} a_{44}}a_{45}\dots\\
f(5) &= 0.a_{51}a_{52}a_{53}a_{54}{\color{red} a_{55}}\dots\\
\vdots 
\end{align*}
\end{frame}

\begin{frame}{Cantor diagonalization}
Now define $b_1,b_2,b_3\dots\in \{0,9\}$ by
$$b_j = \left\lbrace
\begin{array}{cc}
1, & a_{jj} = 0\\
0, & a_{jj} \neq 0
\end{array}\right.$$
\pause
$$b = 0.{\color{blue}{b_1b_2b_3b_4b_5\dots}}$$
\end{frame}

\begin{frame}{Cantor diagonalization}
\begin{itemize}
\item $b\in (0,1)$
\pause
\item $f$ is surjective
\pause
\item there exists $n\in\mathbb{Z}_+$ with $f(n) = b$
\pause
$$0.b_1b_2b_3b_4b_5\dots = 0.a_{n1}a_{n2}a_{n3}a_{n4}a_{n5}\dots$$
\pause
\item $b_n = a_{nn}$ \pause CONTRADICTION!
\end{itemize}
\end{frame}

\begin{frame}{Countable and uncountable}
A set is
\begin{itemize}
\pause
\item \textbf{countably infinite} if it has the same cardinality as $\mathbb{Z}_+$
\pause
\item \textbf{countable} if it is finite or countably infinite
\pause
\item \textbf{uncountable} otherwise
\end{itemize}
\pause
\begin{thm}[Cantor's Theorem]
$\mathbb{R}$ is uncountable
\end{thm}
\end{frame}

\begin{frame}{Ultimate Cantor diagonalization}
The \textbf{power set}  $\mathcal{P}(A)$ of a set $A$ is
$$\mathcal P(A) = \{S: S\subseteq A\}.$$
\pause
\begin{thm}
The cardinality of $\mathcal P(A)$ is strictly larger than $A$.
\end{thm}
\pause
\begin{proof}
\pause
Suppose $f: A\rightarrow \mathcal P(A)$ is surjective.\\
\pause
Consider the set
$$S = \{a\in A: a\notin f(a)\}.$$
\end{proof}
\end{frame}

\begin{frame}{Challenge}
\pause
Since $f$ is surjectve, $S = f(x)$ for some $x\in A$.
\pause
\begin{prob}
Consider the statement $x\in S$.  What can you conclude?
\end{prob}
\end{frame}


\subsection{Set algebra}

\begin{frame}{Unions of sets}
Unions:
\pause
$$A\cup B = \{x: x\in A\ \ \text{or}\ \ x\in B\},$$
\pause
$$A\cup B\cup C = \{x: x\in A\ \ \text{or}\ \ x\in B\ \ \text{or}\ \ x\in C\},$$
\pause
$$A_1\cup A_2\cup \dots \cup A_n = \{x: x\in A_i\ \ \text{for some $1\leq i\leq n$}\}$$
\pause
{\color{red}NOTATION:}
$$\bigcup_{i=1}^n A_i = A_1\cup A_2\cup \dots \cup A_n.$$
\end{frame}

\begin{frame}{Intersections of sets}
Intersections:
\pause
$$A\cap B = \{x: x\in A\ \ \text{and}\ \ x\in B\},$$
\pause
$$A\cap B\cap C = \{x: x\in A\ \ \text{and}\ \ x\in B\ \ \text{and}\ \ x\in C\},$$
\pause
$$A_1\cap A_2\cap \dots \cap A_n = \{x: x\in A_i\ \ \text{for some $1\leq i\leq n$}\}$$
\pause
{\color{red}NOTATION:}
$$\bigcap_{i=1}^n A_i = A_1\cap A_2\cap \dots \cap A_n.$$
\end{frame}

\begin{frame}{Complements of sets}
\pause
The \vocab{complement} of $A$ relative to $B$ is
$$B-A = \{b\in B: b\notin A\}.$$
\pause
\begin{thm}{De Morgan's Laws}
\pause
$$B-\bigcup_{i=1}^n A_i = \bigcap_{i=1}^n (B-A_i)$$
\pause
$$B-\bigcap_{i=1}^n A_i = \bigcup_{i=1}^n (B-A_i)$$
\end{thm}
\end{frame}


\begin{frame}{Families of sets}
A \textbf{family of sets} is a collection
$$\{A_i; i\in I\}$$
\begin{itemize}
\pause
\item $I$ any set, called the \textbf{index set}
\pause
\item $A_i$ is a set for all $i$
\end{itemize}
\pause
$$\bigcup_{i\in I} A_i = \{x: x\in A_i,\ \text{for some $i\in I$}\}$$
\pause
$$\bigcap_{i\in I} A_i = \{x: x\in A_i,\ \text{for all $i\in I$}\}$$
\end{frame}

\begin{frame}{Challenge}
\begin{itemize}
\item index set $I = \mathbb{Z}_+$
\item family of sets $\{A_i: i\in I\}$
\item $A_i = [0,i)$
\end{itemize}
\begin{prob}
Determine $\bigcup_{i\in I} A_i$.
\end{prob}
\end{frame}

\begin{frame}{Challenge}
\begin{itemize}
\item index set $I = \mathbb{Z}_+$
\item family of sets $\{A_i: i\in I\}$
\item $A_i = [0,i)$
\end{itemize}
\begin{prob}
Determine $\bigcap_{i\in I} A_i$.
\end{prob}
\end{frame}


\end{document}


