% !TEX TS-program = pdflatex
% !TEX encoding = UTF-8 Unicode

% This file is a template using the "beamer" package to create slides for a talk or presentation
% - Giving a talk on some subject.
% - The talk is between 15min and 45min long.
% - Style is ornate.

% MODIFIED by Jonathan Kew, 2008-07-06
% The header comments and encoding in this file were modified for inclusion with TeXworks.
% The content is otherwise unchanged from the original distributed with the beamer package.

\documentclass{beamer}
\setbeamercovered{invisible}



% Copyright 2004 by Till Tantau <tantau@users.sourceforge.net>.
%
% In principle, this file can be redistributed and/or modified under
% the terms of the GNU Public License, version 2.
%
% However, this file is supposed to be a template to be modified
% for your own needs. For this reason, if you use this file as a
% template and not specifically distribute it as part of a another
% package/program, I grant the extra permission to freely copy and
% modify this file as you see fit and even to delete this copyright
% notice. 


\mode<presentation>
{
  \usetheme{Warsaw}
  % or ...

  % or whatever (possibly just delete it)
}


\usepackage[english]{babel}
% or whatever

\usepackage[utf8]{inputenc}
% or whatever

\usepackage{times}
\usepackage[T1]{fontenc}
% Or whatever. Note that the encoding and the font should match. If T1
% does not look nice, try deleting the line with the fontenc.

%%% MATH RELATED


%%% AMS math stuff
\usepackage{amsmath}
\usepackage{amssymb}
\usepackage{amsthm}
\usepackage{mathrsfs}
\usepackage{enumerate}

%%% Theorem environments
\newtheorem{thm}{Theorem}[subsection]
\newtheorem{prop}[thm]{Proposition}
\newtheorem{cor}{Corollary}[thm]
\newtheorem{por}[thm]{Porism}
\newtheorem{lem}[thm]{Lemma}
\theoremstyle{definition}
\newtheorem{prob}[thm]{Problem}
\newtheorem{soln}{Solution}
\newtheorem{defn}[thm]{Definition}
\newtheorem{ex}[thm]{Example}
\newtheorem{quest}[thm]{Question}
\newtheorem{remk}[thm]{Remark}

%%% Typesetting shortcuts
\newcommand{\tn}[1]{\textnormal{#1}}
\newcommand{\ol}[1]{\overline{#1}}
\newcommand{\wt}[1]{\widetilde{#1}}
\newcommand{\wh}[1]{\widehat{#1}}
\newcommand{\vocab}[1]{\textbf{#1}\index{#1}}

%%% Math shortcuts
\newcommand{\bbr}{\mathbb R}
\newcommand{\bbz}{\mathbb Z}
\newcommand{\bbq}{\mathbb Q}
\newcommand{\bbn}{\mathbb N}
\newcommand{\bbf}{\mathbb F}
\newcommand{\bbc}{\mathbb C}
\newcommand{\bbd}{\mathbb D}
\newcommand{\bba}{\mathbb A}
\newcommand{\bbp}{\mathbb P}
\newcommand{\bbg}{\mathbb G}
\newcommand{\bbv}{\mathbb V}
\newcommand{\dih}[1]{\mathcal D_{#1}}
\newcommand{\sym}[1]{\mathcal S_{#1}}
\newcommand{\vspan}{\tn{span}}
\newcommand{\trace}{\tn{trace}}
\newcommand{\diff}{\backslash}
\newcommand{\stab}{\tn{stab}}
\newcommand{\conv}{\tn{conv}}
\newcommand{\img}{\tn{img}}
\newcommand{\coker}{\tn{coker}}
\newcommand{\id}{\tn{id}}
\newcommand{\Hom}{\tn{Hom}}
\newcommand{\End}{\tn{End}}
\newcommand{\Aut}{\tn{Aut}}
\newcommand{\aut}{\tn{Aut}}
\newcommand{\ann}{\tn{Ann}}
\newcommand{\GL}{\tn{GL}}
\newcommand{\Gr}{\tn{Gr}}
\newcommand{\lord}{\preccurlyeq}
\newcommand{\rord}{\succcurlyeq}
\newcommand{\tr}{\textnormal{Tr}}
\newcommand{\Tr}{\textnormal{Tr}}
\newcommand{\bbl}{\mathbb{L}}
\newcommand{\C}{\mathscr{C}}
\newcommand{\X}{\mathscr{X}}
\renewcommand{\S}{\mathscr{S}}
\newcommand{\M}{\mathscr{M}}
\renewcommand{\L}{\mathcal{L}}


%%% Algebraic Geometry
\newcommand{\height}{\textnormal{ht}}
\newcommand{\A}{\mathbb{A}}
\newcommand{\p}{\mathfrak{p}}
\newcommand{\sheaf}[1]{\mathcal{#1}}
\newcommand{\spec}{\textnormal{Spec}}
\newcommand{\proj}{\textnormal{Proj}}
\newcommand{\Aff}{\textnormal{Aff}}
\newcommand{\skel}{\textnormal{skel}}
\newcommand{\supp}{\textnormal{supp}}
\newcommand{\orb}{\textnormal{orb}}
\newcommand{\Proj}{\textnormal{Proj}}
\newcommand{\Pic}{\textnormal{Pic}}
\newcommand{\Rees}{\textnormal{Rees}}
\newcommand{\shom}{\mathcal{H}om}

\renewcommand*\arraystretch{1.3}

%%% paper specific definitions
\newcommand{\weyl}{\Omega}
\newcommand{\weyll}{{\widehat{\Omega}}}
\newcommand{\weylll}{{\widetilde{\Omega}}}
\newcommand{\mweyl}{{M_N(\Omega)}}
\newcommand{\mweyll}{{M_N(\widehat{\Omega})}}
\newcommand{\mweylll}{{M_N(\widetilde{\Omega})}}
\newcommand{\seq}{\text{Seq}}
\newcommand{\tail}{\text{Tail}}
\newcommand{\Ad}{\textnormal{Ad}}
\newcommand{\sech}{\textnormal{sech}}
\newcommand{\colim}{\varinjlim}
\newcommand{\limit}{\varprojlim}
\newcommand{\Bis}{\textnormal{Bis}}
\newcommand{\m}{\mathfrak{m}}
\newcommand{\mxx}[4]{\left(\begin{array}{cc} #1 & #2\\ #3 & #4 \end{array}\right)}
\newcommand{\diag}{\text{diag}}
\newcommand{\qdet}{\textnormal{qdet}}
\newcommand{\mdet}{\textnormal{mdet}}
\newcommand{\mtau}{\mathcal{T}}
\newcommand{\cof}{\textnormal{cof}}
\newcommand{\minor}{\textnormal{minor}}
\newcommand{\holo}{Holo}
\newcommand{\ord}{\textnormal{order}}
\newcommand{\mult}{\mathfrak M}






\title{MATH 350-2 Advanced Calculus} 
\subtitle
{} % (optional)

\author[W.R. Casper] % (optional, use only with lots of authors)
{W.R. Casper}
% - Use the \inst{?} command only if the authors have different
%   affiliation.

\institute[California State University Fullerton] % (optional, but mostly needed)
{
  Department of Mathematics\\
  California State University Fullerton}
% - Use the \inst command only if there are several affiliations.
% - Keep it simple, no one is interested in your street address.

\subject{Talks}
% This is only inserted into the PDF information catalog. Can be left
% out. 



% If you have a file called "university-logo-filename.xxx", where xxx
% is a graphic format that can be processed by latex or pdflatex,
% resp., then you can add a logo as follows:

% \pgfdeclareimage[height=0.5cm]{university-logo}{university-logo-filename}
% \logo{\pgfuseimage{university-logo}}



% Delete this, if you do not want the table of contents to pop up at
% the beginning of each subsection:
\AtBeginSubsection[]
{
  \begin{frame}<beamer>{Outline}
    \tableofcontents[currentsection,currentsubsection]
  \end{frame}
}


% If you wish to uncover everything in a step-wise fashion, uncomment
% the following command: 

%\beamerdefaultoverlayspecification{<+->}


\begin{document}

\begin{frame}
  \titlepage
\end{frame}

\begin{frame}{Outline}
  \tableofcontents
  % You might wish to add the option [pausesections]
\end{frame}

% Since this a solution template for a generic talk, very little can
% be said about how it should be structured. However, the talk length
% of between 15min and 45min and the theme suggest that you stick to
% the following rules:  

% - Exactly two or three sections (other than the summary).
% - At *most* three subsections per section.
% - Talk about 30s to 2min per frame. So there should be between about
%   15 and 30 frames, all told.

\section{Real Analysis Lecture 5}

\subsection{Sets, Relations, Functions}

\begin{frame}{Set basics}
Intuitively, a \textbf{set} $A$ is a "collection of things" which we call the \textbf{elements} of $A$.
\begin{itemize}
\pause
\item in practice, this is a bad definition (Russell's Paradox)
\pause
\item true set formulation: Zermelo-Frankel Axioms
\end{itemize}
\pause
Examples:
\begin{itemize}
\pause
\item $\mathbb R$, $\mathbb{Z}_+$, $\mathbb{Z}$, $\mathbb{Q}$
\pause
\item $(1,5]$, $(0,\infty)$
\pause
\item empty set $\varnothing$
\pause
\item $\{\heartsuit,\text{Fall}, \{\varnothing\}\}$
\pause
\item $\{n\in \mathbb{Z}: \text{$n$ is prime}\}$
\end{itemize}
\end{frame}

\begin{frame}{Everything is a set}
In the true minimalistic philosophy of mathematics, we want everything to be a set.
\begin{itemize}
\pause
\item integers
\pause
$$0 = \varnothing, 1 = \{\varnothing\}, 2 = \{\varnothing,\{\varnothing\}\},\dots$$
\pause
\item ordered pairs
\pause
$$(a,b) = \{a,\{a,b\}\}$$
\pause
\item even relations and functions are sets!
\end{itemize}
\end{frame}

\begin{frame}{Challenge!}
\begin{prob}
Prove that for ordered pairs $(a,b)$ and $(c,d)$ that
$$(a,b)=(c,d)\quad\text{if and only if}\quad a=c\ \text{and}\ b=d$$
\end{prob}
\pause
\begin{soln}
$(a,b)=(c,d)$ if and only if $\{a,\{a,b\}\}=\{c,\{c,d\}\}$\\
\pause
Clearly, if $a=c$ and $b=d$, then $\{a,\{a,b\}\}=\{c,\{c,d\}\}$\\
\pause
The tough part is the opposite direction!\\
\pause
Suppose $\{a,\{a,b\}\}=\{c,\{c,d\}\}$.\\
\pause
Two possible cases:
\pause
\begin{align*}
\text{Case I:}\quad a=c\ \ \text{and}\ \  \{a,b\} = \{c,d\}\\
\pause
\text{Case II:}\quad a=\{c,d\}\ \ \text{and}\ \ \{a,b\} = c
\end{align*}
\end{soln}
\end{frame}

\begin{frame}{Challenge!}
\begin{prob}
Prove that for ordered pairs $(a,b)$ and $(c,d)$ that
$$(a,b)=(c,d)\quad\text{if and only if}\quad a=c\ \text{and}\ b=d$$
\end{prob}
\begin{soln}
$$\text{Case I:}\quad a=c\ \ \text{and}\ \  \{a,b\} = \{c,d\}$$
\pause
Since $\{a,b\} = \{c,d\}$, we know $b\in \{c,d\}$.\\
\pause
Therefore $b=c$ or $b=d$.\\
If $b=d$, we're done! \pause ... so assume instead that $b=c$\\
\pause
Then $a=c$ implies $a=b$.\\
\pause
Therefore $\{c,d\} = \{a,b\} = \{a,a\} = \{a\}$.\\
\pause
It follows that $d=c=b=a$.
\end{soln}
\end{frame}

\begin{frame}{Challenge!}
\begin{prob}
Prove that for ordered pairs $(a,b)$ and $(c,d)$ that
$$(a,b)=(c,d)\quad\text{if and only if}\quad a=c\ \text{and}\ b=d$$
\end{prob}
\begin{soln}
$$\text{Case II:}\quad a=\{c,d\}\ \ \text{and}\ \ \{a,b\} = c$$
\pause
This would imply that $c\in a$ and $a\in c$.\\
\pause
This can be shown to contradict the ZF Axioms of Set Theory.\\
\pause
Specifically the regularity axiom for the set $\{a,c\}$...
\end{soln}
\end{frame}

\begin{frame}{Relations}
The \textbf{Cartesian product} of $A$ and $B$ is
$$A\times B = \{(a,b): a\in A,\ b\in B\}.$$
\pause
A \textbf{relation} $\mathcal R$ from $A$ to $B$ is a subset of $A\times B$.\\
\pause
$$\text{\color{red}NOTATION:}\quad a\mathcal R b\quad\text{means}\quad (a,b)\in \mathcal R.$$
\pause
Domain and codomain:
\pause
\begin{align*}
\pause
\text{dom}(\mathcal{R}) &= \{a\in A: \exists b\in B,\ a\mathcal R b\}\\
\text{codom}(\mathcal{R}) &= \{b\in B: \exists a\in A,\ a\mathcal R b\}
\end{align*}
\end{frame}

\begin{frame}{Relations}
A relation $\mathcal R$ from $A$ to $A$ is called a \textbf{relation on} $A$.  A relation on $A$ is
\begin{itemize}
\pause
\item \textbf{reflexive} if $a\mathcal Ra$ for all $a\in A$
\pause
\item \textbf{symmetric} if $a\mathcal Rb$ implies $b\mathcal Ra$ for all $a,b\in A$
\pause
\item \textbf{transitive} if $a\mathcal Rb$ and $b\mathcal Rc$ implies $a\mathcal Rc$ for all $a,b,c\in\mathcal A$
\end{itemize}
\pause
An \textbf{equivalence relation} satisfies all three properties.\\
\pause
Examples:
\pause
\begin{itemize}
\pause
\item $<\ = \{(x,y): y-x\in (0,\infty)\}$ is transitive but not reflexive or symmetric on $\mathbb{R}$
\pause
\item $\leq\ = \{(x,y): y-x\in [0,\infty)\}$ is reflexive and transitive but not symmetric on $\mathbb{R}$
\end{itemize}
\end{frame}

\begin{frame}{Challenge!}
\begin{prob}
Give an example of a relation on $\mathbb{R}$ which is symmetric and transitive but not reflexive.
\end{prob}
\end{frame}

\begin{frame}{Challenge!}
\begin{prob}
Give an example of a relation on $\mathbb{R}$ which is reflexive and symmetric but not transitive.
\end{prob}
\end{frame}

\begin{frame}{Functions}
A \textbf{function} from $A$ to $B$ is a relation $\mathcal R$ from $A$ to $B$ with the property
\pause
$$a\mathcal Rb\ \text{and}\ a\mathcal Rc\quad\text{implies}\quad b=c.$$
\pause
If a relation $\mathcal R$ is a function, we usually use a symbol like $f$.\\
\pause
\begin{align*}
\text{\color{red}NOTATION:}&\quad f: A\rightarrow B\quad\text{means $f$ is a function from $A$ to $B$}\\
\text{\color{red}NOTATION:}&\quad f(a) = b\quad\text{means}\quad (a,b)\in f.
\end{align*}
\pause
The set $$\img(f) = \{f(a): a\in A\}$$ is called the \textbf{range} or \textbf{image} of $f$
\end{frame}

\begin{frame}{Challenge!}
\begin{prob}
Determine all the equivalence relations on $\mathbb{R}$ which are also functions.
\end{prob}
\pause
\begin{soln}
$f=\mathcal R$ must be reflexive, so $x\mathcal Rx$ for all $x$\\
\pause
This means $f(x) = x$ for all $x$.\\
\pause
Thus the only function which is an equivalence relation is the identity function
$$f(x) = x.$$
\end{soln}
\end{frame}

\begin{frame}{Properties of functions}
A function $f$ from $A$ to $B$ is called \textbf{one-to-one} or \textbf{injective} if 
\pause
$$f(x) = f(y) \quad\text{implies}\quad x=y\ \ \text{for all $x,y\in A$}.$$
\pause
It is called \textbf{onto} or \textbf{surjective} if $\img(f) = B$, or equivalently
$$\text{for all $b\in B$ there exists $a\in A$ with $f(a) =b$}.$$
\pause
If it satisfies both properties, it is called \textbf{bijective}.
\end{frame}

\begin{frame}{Inverses}
The \textbf{converse} of a relation $\mathcal R$ from $A$ to $B$ is the relation
\pause
$$\check{\mathcal R} = \{(b,a)\in B\times A: a\mathcal R b\}.$$
\pause
If $f$ is  function, then $\check f$ may or not be a function.\\
\pause
If it is, we call it the \vocab{inverse} of $f$.
\end{frame}

\begin{frame}{Compositions}
The composition of $f: A\rightarrow B$ and $g: B\rightarrow C$ is the function
\pause
$$g\circ f: A\rightarrow C$$
\pause
with domain $A$ and codomain $C$ defined by
\pause
$$(g\circ f)(x)  = g(f(x)).$$
\end{frame}

\begin{frame}{Sequences}
A \textbf{finite sequence} is a function $f: \{1,2,\dots,n\}\rightarrow \mathbb{R}$.\\
\pause
An \textbf{infinite sequence} is a function $f: \mathbb{Z}_+\rightarrow \mathbb{R}$.\\
\pause
$$\text{\color{red}NOTATION:}\quad f_n\quad\text{indicates the value}\ f(n)$$
\pause
$$\text{\color{red}NOTATION:}\quad \{f_n\}\quad\text{is another way of writing the function}\ f$$
\pause
If $k:\mathbb{Z}_+\rightarrow\mathbb{Z}_+$ is a function which is \vocab{strictly increasing}, meaning
$$m < n\quad\Rightarrow\quad k(m) < k(n),$$
\pause
then the composition $f\circ k: \mathbb{Z}_+\rightarrow \mathbb{R}$ forms a sequence called a \vocab{subsequence} of $f$.
\pause
$$\text{\color{red}NOTATION:}\quad \{f_{k(n)}\}\ \text{or}\ \{f_{k_n}\}\quad\text{both really mean}\ f\circ k$$
\end{frame}

\subsection{Cardinality}

\begin{frame}{Cantor's Paradise}
\pause
Two sets $A$ and $B$ have the same \vocab{cardinality} if there is a bijection
\pause
$$f: A\rightarrow B.$$
\pause
$$\text{\color{red}NOTATION:}\quad \lvert A\rvert = \lvert B\rvert.$$
\pause
\begin{thm}[Cantor-Schroeder-Bernstein Theorem]
If there exists an injection $f: A\rightarrow B$ and an injection $g: B\rightarrow A$,
then there exists a bijection $h: A\rightarrow B.$
\end{thm}
\pause
$$\text{\color{red}NOTATION:}\quad \lvert A\rvert \leq \lvert B\rvert\quad\text{means there is an injection from $A$ to $B$}.$$

\end{frame}

\begin{frame}{Finite and infinite}
\pause
Sets with finite cardinality:
$$\{1,2,3\},\ \ \ \{\mathbb{Z},\mathbb{R}\},\ \ \ \{x:\ \text{$x$ is a student at CSUF}\}$$
\pause
Sets with infinite cardinality:
$$\mathbb{Z}_+,\ \ \ \mathbb{Q},\ \ \ \mathbb{R},\ \ \ \mathbb{C}.$$
\pause
Cantor's discovery: there are multiple sizes of infinity!
$$\mathbb{Z}_+,\ \mathbb{Z},\ \mathbb{Z}\times\mathbb{Z},\ \mathbb{Q},\ \text{are all the same cardinality}$$
\pause
$$\mathbb{R}\  \text{has larger cardinality than}\ \mathbb{Z}_+.$$
\end{frame}


\end{document}


