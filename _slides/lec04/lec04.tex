% !TEX TS-program = pdflatex
% !TEX encoding = UTF-8 Unicode

% This file is a template using the "beamer" package to create slides for a talk or presentation
% - Giving a talk on some subject.
% - The talk is between 15min and 45min long.
% - Style is ornate.

% MODIFIED by Jonathan Kew, 2008-07-06
% The header comments and encoding in this file were modified for inclusion with TeXworks.
% The content is otherwise unchanged from the original distributed with the beamer package.

\documentclass{beamer}
\setbeamercovered{invisible}



% Copyright 2004 by Till Tantau <tantau@users.sourceforge.net>.
%
% In principle, this file can be redistributed and/or modified under
% the terms of the GNU Public License, version 2.
%
% However, this file is supposed to be a template to be modified
% for your own needs. For this reason, if you use this file as a
% template and not specifically distribute it as part of a another
% package/program, I grant the extra permission to freely copy and
% modify this file as you see fit and even to delete this copyright
% notice. 


\mode<presentation>
{
  \usetheme{Warsaw}
  % or ...

  % or whatever (possibly just delete it)
}


\usepackage[english]{babel}
% or whatever

\usepackage[utf8]{inputenc}
% or whatever

\usepackage{times}
\usepackage[T1]{fontenc}
% Or whatever. Note that the encoding and the font should match. If T1
% does not look nice, try deleting the line with the fontenc.

%%% MATH RELATED


%%% AMS math stuff
\usepackage{amsmath}
\usepackage{amssymb}
\usepackage{amsthm}
\usepackage{mathrsfs}
\usepackage{enumerate}

%%% Theorem environments
\newtheorem{thm}{Theorem}[subsection]
\newtheorem{prop}[thm]{Proposition}
\newtheorem{cor}{Corollary}[thm]
\newtheorem{por}[thm]{Porism}
\newtheorem{lem}[thm]{Lemma}
\theoremstyle{definition}
\newtheorem{prob}[thm]{Problem}
\newtheorem{soln}{Solution}
\newtheorem{defn}[thm]{Definition}
\newtheorem{ex}[thm]{Example}
\newtheorem{quest}[thm]{Question}
\newtheorem{remk}[thm]{Remark}

%%% Typesetting shortcuts
\newcommand{\tn}[1]{\textnormal{#1}}
\newcommand{\ol}[1]{\overline{#1}}
\newcommand{\wt}[1]{\widetilde{#1}}
\newcommand{\wh}[1]{\widehat{#1}}
\newcommand{\vocab}[1]{\textbf{#1}\index{#1}}

%%% Math shortcuts
\newcommand{\bbr}{\mathbb R}
\newcommand{\bbz}{\mathbb Z}
\newcommand{\bbq}{\mathbb Q}
\newcommand{\bbn}{\mathbb N}
\newcommand{\bbf}{\mathbb F}
\newcommand{\bbc}{\mathbb C}
\newcommand{\bbd}{\mathbb D}
\newcommand{\bba}{\mathbb A}
\newcommand{\bbp}{\mathbb P}
\newcommand{\bbg}{\mathbb G}
\newcommand{\bbv}{\mathbb V}
\newcommand{\dih}[1]{\mathcal D_{#1}}
\newcommand{\sym}[1]{\mathcal S_{#1}}
\newcommand{\vspan}{\tn{span}}
\newcommand{\trace}{\tn{trace}}
\newcommand{\diff}{\backslash}
\newcommand{\stab}{\tn{stab}}
\newcommand{\conv}{\tn{conv}}
\newcommand{\img}{\tn{img}}
\newcommand{\coker}{\tn{coker}}
\newcommand{\id}{\tn{id}}
\newcommand{\Hom}{\tn{Hom}}
\newcommand{\End}{\tn{End}}
\newcommand{\Aut}{\tn{Aut}}
\newcommand{\aut}{\tn{Aut}}
\newcommand{\ann}{\tn{Ann}}
\newcommand{\GL}{\tn{GL}}
\newcommand{\Gr}{\tn{Gr}}
\newcommand{\lord}{\preccurlyeq}
\newcommand{\rord}{\succcurlyeq}
\newcommand{\tr}{\textnormal{Tr}}
\newcommand{\Tr}{\textnormal{Tr}}
\newcommand{\bbl}{\mathbb{L}}
\newcommand{\C}{\mathscr{C}}
\newcommand{\X}{\mathscr{X}}
\renewcommand{\S}{\mathscr{S}}
\newcommand{\M}{\mathscr{M}}
\renewcommand{\L}{\mathcal{L}}


%%% Algebraic Geometry
\newcommand{\height}{\textnormal{ht}}
\newcommand{\A}{\mathbb{A}}
\newcommand{\p}{\mathfrak{p}}
\newcommand{\sheaf}[1]{\mathcal{#1}}
\newcommand{\spec}{\textnormal{Spec}}
\newcommand{\proj}{\textnormal{Proj}}
\newcommand{\Aff}{\textnormal{Aff}}
\newcommand{\skel}{\textnormal{skel}}
\newcommand{\supp}{\textnormal{supp}}
\newcommand{\orb}{\textnormal{orb}}
\newcommand{\Proj}{\textnormal{Proj}}
\newcommand{\Pic}{\textnormal{Pic}}
\newcommand{\Rees}{\textnormal{Rees}}
\newcommand{\shom}{\mathcal{H}om}

\renewcommand*\arraystretch{1.3}

%%% paper specific definitions
\newcommand{\weyl}{\Omega}
\newcommand{\weyll}{{\widehat{\Omega}}}
\newcommand{\weylll}{{\widetilde{\Omega}}}
\newcommand{\mweyl}{{M_N(\Omega)}}
\newcommand{\mweyll}{{M_N(\widehat{\Omega})}}
\newcommand{\mweylll}{{M_N(\widetilde{\Omega})}}
\newcommand{\seq}{\text{Seq}}
\newcommand{\tail}{\text{Tail}}
\newcommand{\Ad}{\textnormal{Ad}}
\newcommand{\sech}{\textnormal{sech}}
\newcommand{\colim}{\varinjlim}
\newcommand{\limit}{\varprojlim}
\newcommand{\Bis}{\textnormal{Bis}}
\newcommand{\m}{\mathfrak{m}}
\newcommand{\mxx}[4]{\left(\begin{array}{cc} #1 & #2\\ #3 & #4 \end{array}\right)}
\newcommand{\diag}{\text{diag}}
\newcommand{\qdet}{\textnormal{qdet}}
\newcommand{\mdet}{\textnormal{mdet}}
\newcommand{\mtau}{\mathcal{T}}
\newcommand{\cof}{\textnormal{cof}}
\newcommand{\minor}{\textnormal{minor}}
\newcommand{\holo}{Holo}
\newcommand{\ord}{\textnormal{order}}
\newcommand{\mult}{\mathfrak M}






\title{MATH 350-2 Advanced Calculus} 
\subtitle
{} % (optional)

\author[W.R. Casper] % (optional, use only with lots of authors)
{W.R. Casper}
% - Use the \inst{?} command only if the authors have different
%   affiliation.

\institute[California State University Fullerton] % (optional, but mostly needed)
{
  Department of Mathematics\\
  California State University Fullerton}
% - Use the \inst command only if there are several affiliations.
% - Keep it simple, no one is interested in your street address.

\subject{Talks}
% This is only inserted into the PDF information catalog. Can be left
% out. 



% If you have a file called "university-logo-filename.xxx", where xxx
% is a graphic format that can be processed by latex or pdflatex,
% resp., then you can add a logo as follows:

% \pgfdeclareimage[height=0.5cm]{university-logo}{university-logo-filename}
% \logo{\pgfuseimage{university-logo}}



% Delete this, if you do not want the table of contents to pop up at
% the beginning of each subsection:
\AtBeginSubsection[]
{
  \begin{frame}<beamer>{Outline}
    \tableofcontents[currentsection,currentsubsection]
  \end{frame}
}


% If you wish to uncover everything in a step-wise fashion, uncomment
% the following command: 

%\beamerdefaultoverlayspecification{<+->}


\begin{document}

\begin{frame}
  \titlepage
\end{frame}

\begin{frame}{Outline}
  \tableofcontents
  % You might wish to add the option [pausesections]
\end{frame}

% Since this a solution template for a generic talk, very little can
% be said about how it should be structured. However, the talk length
% of between 15min and 45min and the theme suggest that you stick to
% the following rules:  

% - Exactly two or three sections (other than the summary).
% - At *most* three subsections per section.
% - Talk about 30s to 2min per frame. So there should be between about
%   15 and 30 frames, all told.

\section{Real Analysis Lecture 4}

\subsection{Decimal expansions}

\begin{frame}{Decimal approximations}
A \textbf{finite decimal expansion} is an expression
$$r = a_0 + \frac{a_1}{10} + \frac{a_2}{100} + \frac{a_3}{10^3} + \dots + \frac{a_n}{10^n},$$
where $a_0\in\mathbb Z_+$ and $0\leq a_k\leq 9$ for $1\leq k\leq n$.\\
\pause
\textbf{Notation:}
$$a_0.a_1a_2a_3\dots a_n.$$
\pause
Any positive real number $x>0$ can be approximated by a finite decimal expansion.
\end{frame}

\begin{frame}{Decimal approximations}
\begin{thm}[Apostol Theorem 1.20]
For any real $x > 0$ and $n\in \mathbb{Z}_+$, there exists a finite decimal expansion $r_n=a_0.a_1a_2\dots a_n$ with
$$r_n \leq x < r_n + \frac{1}{10^n}.$$
\end{thm}
\end{frame}

\begin{frame}{Decimal approximations}
\begin{proof}
Consider the set
$$A = \{a\in\mathbb{Z}: a\leq x\}.$$
\pause
The set $A$ is a set of integers which is bounded above by $x$, so it has a maximum $a_0$. \\
\pause
Then clearly $x_1 = a-a_0\in [0,1)$. \\
\pause
Define $a_1,a_2,a_3,\dots$ and $x_1,x_2,x_3,\dots$ recursively by
$$a_k = \max\{a\in\mathbb{Z}_+: a\leq 10x_{k}\}$$
and $x_{k+1}=10x_k-a_k$.
\pause
Then $0\leq a_k \leq 9$ for all $k\geq 1$ 
\pause
and 
$$a_0 + \frac{a_1}{10} + \frac{a_2}{100} + \dots + \frac{a_n}{10^n}
\leq x < a_0 + \frac{a_1}{10} + \frac{a_2}{100} + \dots + \frac{a_n+1}{10^n}
$$
\end{proof}
\end{frame}

\begin{frame}{Decimal expansions}
We say that $x>0$ has the decimal expansion $a_0.a_1a_2a_3\dots$ and write
$$x = a_0.a_1a_2a_3a_4\dots$$
\pause
if for all $n\in\mathbb{Z}_+$,
$$a_0 + \frac{a_1}{10} + \frac{a_2}{100} + \dots + \frac{a_n}{10^n}
\leq x < a_0 + \frac{a_1}{10} + \frac{a_2}{100} + \dots + \frac{a_n+1}{10^n}.
$$
\pause
Note: this is slightly different than the usual limit meaning, for two good reasons:
\begin{enumerate}
\pause
\item we haven't defined limits
\pause
\item with this definition, numbers have unique decimal expansions!
\pause
$$1\neq 0.999999999\dots$$
\pause
$$1 \nless 0 + \frac{9}{10} + \frac{9}{100} + \dots + \frac{9+1}{10^n} = 1.$$
\end{enumerate}
\end{frame}

\begin{frame}{Challenge!}
\begin{prob}
Find the decimal expansion of $1/7$.
\end{prob}
\end{frame}

\begin{frame}{Challenge!}
\begin{prob}
Find a rational number whose decimal expansion is 
$$0.45454545\dots.$$
\end{prob}
\end{frame}

\begin{frame}{Challenge!}
\begin{prob}
Which kinds of numbers have decimal expansions that end?  (Meaning that after a while, all the decimals are zero?
\end{prob}
\end{frame}

\begin{frame}{Challenge!}
\begin{prob}
Which kinds of numbers have decimal expansions that repeat?
\end{prob}
\end{frame}

\subsection{The Triangle Inequality}

\begin{frame}{Absolute value}
The \textbf{absolute value} of $x\in \mathbb{R}$ is
$$\lvert x \rvert = \left\lbrace\begin{array}{cc}
 x,& x \geq 0\\
-x,& x < 0
\end{array}\right.$$
In particular,
$$0\leq \lvert x\rvert$$
and also
$$-\lvert x\rvert\leq x \leq \lvert x\rvert.$$
\end{frame}

\begin{frame}{Challenge!}
\begin{prob}
Prove Apostol Theorem 1.21 that if $a\geq 0$, then $\lvert x\rvert \leq a$ if and only if $-a\leq x\leq a$.
\end{prob}
\pause 
\begin{soln}
If $\lvert x\rvert \leq a$ then $-a\leq -\lvert x\rvert$
\pause 
and therefore
$$-a\leq -\lvert x\rvert \leq x\leq \lvert x\rvert\leq a$$
\pause 
Conversely, if $-a\leq x\leq a$ then
\pause 
$$x\geq 0\Rightarrow \lvert x\rvert = x\leq a$$
\pause 
$$x\leq 0\Rightarrow \lvert x\rvert = -x\leq -(-a) = a$$
\end{soln}
\end{frame}

\begin{frame}{Basic triangle inequality}
\begin{thm}[Triangle inequality]
For any real numbers $x,y\in\mathbb{R}$ we have
$$\lvert x+y\rvert\leq \lvert x \rvert + \lvert y\rvert.$$
\end{thm}
\pause 
\begin{proof}
$$
-\lvert x \rvert \leq x\leq \lvert x\rvert
\quad\text{and}\quad
-\lvert y \rvert \leq y\leq \lvert y\rvert
$$
\pause
Adding these together, we get
$$
-(\lvert x \rvert + \lvert y \rvert) \leq x + y\leq \lvert x\rvert + \lvert y\rvert
$$
\pause
It follows from the previous theorem that
$$\lvert x + y\rvert \leq \lvert x\rvert + \lvert y\rvert$$
\end{proof}
\end{frame}

\begin{frame}{Advanced triangle inequality}
\begin{thm}[Triangle inequality]
For any real numbers $x_1,x_2,\dots,x_n\in\mathbb{R}$ we have
$$\lvert x_1+x_2+\dots+x_n\rvert\leq \lvert x_1 \rvert + \lvert x_2\rvert +  \dots + \lvert x_n\rvert.$$
\end{thm}
\pause
\begin{proof}
Induction.
\end{proof}
\end{frame}

\begin{frame}{Higher-dimensional triangle inequality}
\begin{thm}[Cauchy-Schwartz Inequality (Apostol Theorem 1.23)]
If $x_1,\dots,x_n$ and $y_1,\dots, y_n$ are real numbers, then
$$\left(\sum_{k=1}^n x_ky_k\right)^2\leq \left(\sum_{k=1}^n x_k^2\right)\left(\sum_{k=1}^n y_k^2\right)$$
If $y_k$ isn't always zero, then equality holds if and only if there exists $t\in\mathbb{R}$ with $x_k = ty_k$ for all $k$.
\end{thm}
\pause
Vector version:
$$(\vec x\cdot\vec y)^2 \leq \lvert \vec x\rvert^2\lvert \vec y\rvert^2.$$
\end{frame}

\begin{frame}{Higher-dimensional triangle inequality}
\begin{proof}
\pause
$$\sum_{k=1}^n (x_k + ty_k)^2\geq 0,\quad \text{with equality iff all terms zero}$$
\pause
$$\sum_{k=1}^n x_k^2 + 2t\sum_{k=1}^n x_ky_k + t^2\sum_{k=1}^ny_k^2\geq 0$$
\pause
Take $t = -\left(\sum_{k=1}^n x_ky_k\right)/\left(\sum_{k=1}^ny_k^2\right):$
\pause
$$\sum_{k=1}^n x_k^2 -\frac{ \left(\sum_{k=1}^n x_ky_k\right)^2}{\left(\sum_{k=1}^ny_k^2\right)}\geq 0.$$
\end{proof}
\end{frame}

\begin{frame}{Higher-dimensional triangle inequality}
\begin{thm}[Minkowski inequality]
For any real numbers $x_1,x_2,\dots,x_n\in\mathbb{R}$ and $y_1,y_2,\dots,y_n\in\mathbb{R}$ we have
$$\left(\sum_{k=1}^n (x_k+y_k)^2\right)^{1/2}\leq \left(\sum_{k=1}^n x_k^2\right)^{1/2} + \left(\sum_{k=1}^ny_k^2\right)^{1/2}$$
\end{thm}
\pause
Vector version:
$$\lvert \vec x+\vec y\rvert \leq \lvert \vec x\rvert + \lvert\vec y\rvert.$$
\pause
A higher dimensional triangle inequality!
\end{frame}

\begin{frame}{Higher-dimensional triangle inequality}
\begin{proof}
By the triangle inequality:
\pause
{\tiny
\begin{align*}
\sum_{k=1}^n (x_k+y_k)^2
  & \leq \sum_{k=1}^n (|x_k|+|y_k|)|x_k+y_k|\\
  & =    \sum_{k=1}^n |x_k|\cdot|x_k+y_k| + \sum_{k=1}^n |y_k|\cdot|x_k+y_k|\\
\end{align*}
}
\pause
Now applying the Cauchy-Schwartz inequality: 
\pause
{\tiny
\begin{align*}
\pause
  & \leq \left(\sum_{k=1}^n |x_k|^2\right)^{1/2}\left(\sum_{k=1}^n |x_k+y_k|^2\right)^{1/2}  +  \left(\sum_{k=1}^n |y_k|^2\right)^{1/2}\left(\sum_{k=1}^n |x_k+y_k|^2\right)^{1/2}\\
  & = \left[\left(\sum_{k=1}^n |x_k|^2\right)^{1/2}  +  \left(\sum_{k=1}^n |y_k|^2\right)^{1/2}\right] \left(\sum_{k=1}^n |x_k+y_k|^2\right)^{1/2}
\end{align*}
}
\end{proof}
\end{frame}

\end{document}


