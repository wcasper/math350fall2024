% !TEX TS-program = pdflatex
% !TEX encoding = UTF-8 Unicode

% This file is a template using the "beamer" package to create slides for a talk or presentation
% - Giving a talk on some subject.
% - The talk is between 15min and 45min long.
% - Style is ornate.

% MODIFIED by Jonathan Kew, 2008-07-06
% The header comments and encoding in this file were modified for inclusion with TeXworks.
% The content is otherwise unchanged from the original distributed with the beamer package.

\documentclass{beamer}
\setbeamercovered{invisible}



% Copyright 2004 by Till Tantau <tantau@users.sourceforge.net>.
%
% In principle, this file can be redistributed and/or modified under
% the terms of the GNU Public License, version 2.
%
% However, this file is supposed to be a template to be modified
% for your own needs. For this reason, if you use this file as a
% template and not specifically distribute it as part of a another
% package/program, I grant the extra permission to freely copy and
% modify this file as you see fit and even to delete this copyright
% notice. 


\mode<presentation>
{
  \usetheme{Warsaw}
  % or ...

  % or whatever (possibly just delete it)
}


\usepackage[english]{babel}
% or whatever

\usepackage[utf8]{inputenc}
% or whatever

\usepackage{times}
\usepackage[T1]{fontenc}
% Or whatever. Note that the encoding and the font should match. If T1
% does not look nice, try deleting the line with the fontenc.

%%% MATH RELATED


%%% AMS math stuff
\usepackage{amsmath}
\usepackage{amssymb}
\usepackage{amsthm}
\usepackage{mathrsfs}
\usepackage{enumerate}

%%% Theorem environments
\newtheorem{thm}{Theorem}[subsection]
\newtheorem{prop}[thm]{Proposition}
\newtheorem{cor}{Corollary}[thm]
\newtheorem{por}[thm]{Porism}
\newtheorem{lem}[thm]{Lemma}
\theoremstyle{definition}
\newtheorem{prob}[thm]{Problem}
\newtheorem{soln}{Solution}
\newtheorem{defn}[thm]{Definition}
\newtheorem{ex}[thm]{Example}
\newtheorem{quest}[thm]{Question}
\newtheorem{remk}[thm]{Remark}

%%% Typesetting shortcuts
\newcommand{\tn}[1]{\textnormal{#1}}
\newcommand{\ol}[1]{\overline{#1}}
\newcommand{\wt}[1]{\widetilde{#1}}
\newcommand{\wh}[1]{\widehat{#1}}
\newcommand{\vocab}[1]{\textbf{#1}\index{#1}}

%%% Math shortcuts
\newcommand{\bbr}{\mathbb R}
\newcommand{\bbz}{\mathbb Z}
\newcommand{\bbq}{\mathbb Q}
\newcommand{\bbn}{\mathbb N}
\newcommand{\bbf}{\mathbb F}
\newcommand{\bbc}{\mathbb C}
\newcommand{\bbd}{\mathbb D}
\newcommand{\bba}{\mathbb A}
\newcommand{\bbp}{\mathbb P}
\newcommand{\bbg}{\mathbb G}
\newcommand{\bbv}{\mathbb V}
\newcommand{\dih}[1]{\mathcal D_{#1}}
\newcommand{\sym}[1]{\mathcal S_{#1}}
\newcommand{\vspan}{\tn{span}}
\newcommand{\trace}{\tn{trace}}
\newcommand{\diff}{\backslash}
\newcommand{\stab}{\tn{stab}}
\newcommand{\conv}{\tn{conv}}
\newcommand{\img}{\tn{img}}
\newcommand{\coker}{\tn{coker}}
\newcommand{\id}{\tn{id}}
\newcommand{\Hom}{\tn{Hom}}
\newcommand{\End}{\tn{End}}
\newcommand{\Aut}{\tn{Aut}}
\newcommand{\aut}{\tn{Aut}}
\newcommand{\ann}{\tn{Ann}}
\newcommand{\GL}{\tn{GL}}
\newcommand{\Gr}{\tn{Gr}}
\newcommand{\lord}{\preccurlyeq}
\newcommand{\rord}{\succcurlyeq}
\newcommand{\tr}{\textnormal{Tr}}
\newcommand{\Tr}{\textnormal{Tr}}
\newcommand{\bbl}{\mathbb{L}}
\newcommand{\C}{\mathscr{C}}
\newcommand{\X}{\mathscr{X}}
\renewcommand{\S}{\mathscr{S}}
\newcommand{\M}{\mathscr{M}}
\renewcommand{\L}{\mathcal{L}}


%%% Algebraic Geometry
\newcommand{\height}{\textnormal{ht}}
\newcommand{\A}{\mathbb{A}}
\newcommand{\p}{\mathfrak{p}}
\newcommand{\sheaf}[1]{\mathcal{#1}}
\newcommand{\spec}{\textnormal{Spec}}
\newcommand{\proj}{\textnormal{Proj}}
\newcommand{\Aff}{\textnormal{Aff}}
\newcommand{\skel}{\textnormal{skel}}
\newcommand{\supp}{\textnormal{supp}}
\newcommand{\orb}{\textnormal{orb}}
\newcommand{\Proj}{\textnormal{Proj}}
\newcommand{\Pic}{\textnormal{Pic}}
\newcommand{\Rees}{\textnormal{Rees}}
\newcommand{\shom}{\mathcal{H}om}

\renewcommand*\arraystretch{1.3}

%%% paper specific definitions
\newcommand{\weyl}{\Omega}
\newcommand{\weyll}{{\widehat{\Omega}}}
\newcommand{\weylll}{{\widetilde{\Omega}}}
\newcommand{\mweyl}{{M_N(\Omega)}}
\newcommand{\mweyll}{{M_N(\widehat{\Omega})}}
\newcommand{\mweylll}{{M_N(\widetilde{\Omega})}}
\newcommand{\seq}{\text{Seq}}
\newcommand{\tail}{\text{Tail}}
\newcommand{\Ad}{\textnormal{Ad}}
\newcommand{\sech}{\textnormal{sech}}
\newcommand{\colim}{\varinjlim}
\newcommand{\limit}{\varprojlim}
\newcommand{\Bis}{\textnormal{Bis}}
\newcommand{\m}{\mathfrak{m}}
\newcommand{\mxx}[4]{\left(\begin{array}{cc} #1 & #2\\ #3 & #4 \end{array}\right)}
\newcommand{\diag}{\text{diag}}
\newcommand{\qdet}{\textnormal{qdet}}
\newcommand{\mdet}{\textnormal{mdet}}
\newcommand{\mtau}{\mathcal{T}}
\newcommand{\cof}{\textnormal{cof}}
\newcommand{\minor}{\textnormal{minor}}
\newcommand{\holo}{Holo}
\newcommand{\ord}{\textnormal{order}}
\newcommand{\mult}{\mathfrak M}






\title{MATH 350-2 Advanced Calculus} 
\subtitle
{} % (optional)

\author[W.R. Casper] % (optional, use only with lots of authors)
{W.R. Casper}
% - Use the \inst{?} command only if the authors have different
%   affiliation.

\institute[California State University Fullerton] % (optional, but mostly needed)
{
  Department of Mathematics\\
  California State University Fullerton}
% - Use the \inst command only if there are several affiliations.
% - Keep it simple, no one is interested in your street address.

\subject{Talks}
% This is only inserted into the PDF information catalog. Can be left
% out. 



% If you have a file called "university-logo-filename.xxx", where xxx
% is a graphic format that can be processed by latex or pdflatex,
% resp., then you can add a logo as follows:

% \pgfdeclareimage[height=0.5cm]{university-logo}{university-logo-filename}
% \logo{\pgfuseimage{university-logo}}



% Delete this, if you do not want the table of contents to pop up at
% the beginning of each subsection:
\AtBeginSubsection[]
{
  \begin{frame}<beamer>{Outline}
    \tableofcontents[currentsection,currentsubsection]
  \end{frame}
}


% If you wish to uncover everything in a step-wise fashion, uncomment
% the following command: 

%\beamerdefaultoverlayspecification{<+->}


\begin{document}

\begin{frame}
  \titlepage
\end{frame}

\begin{frame}{Outline}
  \tableofcontents
  % You might wish to add the option [pausesections]
\end{frame}

% Since this a solution template for a generic talk, very little can
% be said about how it should be structured. However, the talk length
% of between 15min and 45min and the theme suggest that you stick to
% the following rules:  

% - Exactly two or three sections (other than the summary).
% - At *most* three subsections per section.
% - Talk about 30s to 2min per frame. So there should be between about
%   15 and 30 frames, all told.

\section{Real Analysis Lecture 14}

\subsection{More on limits}

\begin{frame}{Existence of limits}
\begin{thm}
If a sequence $\{x_n\}$ in a metric space $(M,d)$ converges to a value $L\in M$, then the range
$$X = \{x_1,x_2,\dots\}\subseteq M$$
is a bounded set and $L$ is an adherent point of $X$.
\end{thm}
\pause
When the range of a sequence is bounded, we call the sequence $\{x_n\}$ bounded.
\end{frame}

\begin{frame}{Existence of limits}
\begin{proof}
Suppose that $\lim_{n\rightarrow\infty} x_n = L$.\\
\pause
Let $\epsilon = 2024$.\\
\pause
Then there exists $N\in\mathbb{Z}_+$ with $d(x_n,L)\leq 2024$ for all $n\geq N$.\\
\pause
If we take
$$R = \max\{d(x_1,L),d(x_2,L),\dots, d(x_{N-1},L),2024\},$$
\pause
then $d(x_n,L)\leq R$ for all $N$.\\
\pause
Therefore $X\subseteq B_M(L,R)$ and in particular $X$ is bounded.
\end{proof}
\end{frame}

\begin{frame}{Existence of limits}
Sort of a converse of the previous theorem.
\pause
\begin{thm}
If $S$ is a subset of a metric space $(M,d)$ and $L\in M$ is an adherent point of $S$, then there exists a sequence $\{x_n\}$ of elements of $S$ which converges to $L$.
\end{thm}
\end{frame}

\begin{frame}{Existence of limits}
\begin{proof}
Suppose $L$ is an adherent point of $S$.\\
\pause
Then for all $n\in\mathbb{Z}_+$, the ball $B_M(x,\frac{1}{n})$ contains at least one element of $S$.\\
\pause
For each $n\in\mathbb{Z}_+$, choose $x_n\in S\cap B_M(x,\frac{1}{n})$.\\
\pause
We claim $\lim x_n = L$.\\
\pause
To see this, let $\epsilon > 0$.\\
\pause
Then choose $N\in\mathbb{Z}_+$ with $N>1/\epsilon$.\\
\pause
It follows that for all $n\geq N$
$$d(x_n,L) < 1/n < 1/N < \epsilon.$$
\pause
Since $\epsilon > 0$ was arbitrary, this proves convergence.
\end{proof}
\end{frame}

\begin{frame}{Existence of limits}
Recall a \textbf{subsequence} of a sequence $\{x_n\}$ is a sequence of the form $\{x_{k(n)}\}$, for $k(1) < k(2) < \dots$ positive integers.
\pause
\begin{thm}
A sequence $\{x_n\}$ in a metric space $(M,d)$ converges to a value $L\in M$ if and only if every subsequence $\{x_{k(n)}\}$ also converges to $L$.
\end{thm}
\end{frame}

\begin{frame}{Existence of limits}
\begin{proof}
Suppose that every subsequence $\{x_{k(n)}\}$ of $\{x_n\}$ converges to $L$.\\
\pause
A sequence is a subsequence of itself!  Thus $\{x_n\}$ will converge to $L$.\\
\pause
Conversely, suppose that $\{x_n\}$ converges to $L$.\\
\pause
Let $\{x_{k(n)}\}$ be a subsequence of $\{x_n\}$.\\
\pause
To show it limits to $L$, let $\epsilon > 0$.\\
\pause
Then there exists $N\in\mathbb{Z}_+$ such that $n\geq N$ implies $d(x_n,L)<\epsilon$.\\
\pause
Moreover, $k(n)\geq n$ so $k(n)\geq N$ and therefore $d(x_{k(n)}, L) < \epsilon$.\\
\pause
Since $\epsilon > 0$ was arbitrary, this proves convergence.
\end{proof}
\end{frame}

\begin{frame}{Monotone sequences}
\begin{defn}
\pause
A sequence $\{x_n\}$ of real numbers is called \textbf{increasing} or \textbf{monotone increasing} if $x_n\leq x_{n+1}$ for all $n$.\\
\pause
It is called \textbf{strictly increasing} if $x_n < x_{n+1}$ for all $n$.\\
\pause
A sequence $\{x_n\}$ of real numbers is called \textbf{decreasing} or \textbf{monotone decreasing} if $x_n\geq x_{n+1}$ for all $n$.\\
\pause
It is called \textbf{strictly decreasing} if $x_n > x_{n+1}$ for all $n$.\\
\pause
In any of these cases, we call the sequence \textbf{monotone}.
\end{defn}
\end{frame}

\begin{frame}{Monotone sequences}
Bounded monotone sequences always converge (with Euclidean metric)!
\pause
\begin{thm}[Monotone convergence theorem]
\pause
Consider the metric space formed by $\mathbb{R}$ with Euclidean distance.\\
\pause
A monotone increasing sequence $\{x_n\}$ of real numbers which is bounded above converges to $L = \sup\{x_1,x_2,\dots\}$.\\
\pause
A monotone decreasing sequence $\{x_n\}$ which is bounded below converges to $L = \inf\{x_1,x_2,\dots\}$.
\end{thm}
\end{frame}

\begin{frame}{Monotone sequences}
\begin{proof}
\pause
Suppose that $\{x_n\}$ is a monotone increasing sequence which is bounded above and let $X = \{x_1,x_2,\dots\}$.
\pause
Then by the Completeness Axiom $L=\sup X$ exists.\\
\pause
Let $\epsilon > 0$.\\
\pause
Then $L-\epsilon$ is not a supremum of $X$, so there exists $x_N\in X$ such that $x_N > L-\epsilon$.\\
\pause
Since the sequence is increasing, for any $n >N$, we have that $L-\epsilon < x_N \leq x_n < L$.\\
\pause
It follows that
$$d(x_n,L) = |x_n-L| = L-x_n < \epsilon.$$
\pause
Since $\epsilon > 0$ was arbitrary, this proves $x_n$ converges to $L$.
\end{proof}
\end{frame}

\subsection{Cauchy sequences}

\begin{frame}{Cauchy sequence}
A sequence $\{x_n\}$ in a metric space $(M,d)$ is called Cauchy if the terms get arbitrarily close together as $n$ gets arbitrarily large.\\
\pause
More formally:
\pause
\begin{defn}
A sequence $\{x_n\}$ is called a \textbf{Cauchy sequence} if
$$\forall \epsilon > 0,\ \text{there exists $N\in\mathbb Z_{+}$ with}\ m,n\geq N\ \Rightarrow\ d(x_m,x_n) < \epsilon.$$
\end{defn}
\begin{itemize}
\pause
\item every convergent sequence is Cauchy
\pause
\item is every Cauchy sequence convergent?
\end{itemize}
\end{frame}

\begin{frame}{Challenge!}
\begin{prob}
Prove that if $\{x_n\}$ is a Cauchy sequence in a metric space $(M,d)$, then it is bounded.
\end{prob}
\pause
\begin{soln}
\pause
Suppose $\{x_n\}$ is Cauchy.\\
\pause
Let $\epsilon = 2024$.\\
\pause
Then $\exists\ N\in\mathbb{Z}_+$ with $d(x_m,x_n) < 2024$ for all $m,n\geq N$.\\
\pause
Therefore for all $n\geq N$, we have $d(x_n,x_N)\leq 2024$.\\
\pause
So if we define
$$R = \max\{d(x_1,x_N), d(x_2,x_N),\dots, d(x_{n-1},x_N),2024\},$$
\pause
then $d(x_n,x_N)\leq 2024$ for all $n\geq 1$.\\
\pause
Hence $X = \{x_1,x_2,\dots\}\subseteq B_M(x_N,R)$, and is bounded.
\end{soln}
\end{frame}

\begin{frame}{Cauchy sequence}
Every convergent sequence is Cauchy.
\pause
\begin{thm}
Let $(M,d)$ be a metric space and suppose that $\{x_n\}$ is a sequence in $M$ which converges to $L\in M$.\\
Then $\{x_n\}$ is Cauchy.
\end{thm}
\pause
\begin{proof}
\pause
Let $\epsilon > 0$.\\
\pause
Then there exists $N$ such that $n\geq N$ implies $d(x_n,L) < \epsilon/2$.\\
\pause
Therefore for any $m,n\geq N$ we have
$$d(x_m,x_n)\leq d(x_m,L) + d(L,x_n) < \epsilon/2 + \epsilon/2 = \epsilon.$$
Since $\epsilon > 0$ was arbitrary, this proves that $\{x_n\}$ is Cauchy.
\end{proof}
\end{frame}

\begin{frame}{Cauchy sequence}
Partial converse.
\pause
\begin{thm}
Let $M=\mathbb R^n$ with the Eucliean metric, and suppose that $\{x_n\}$ is a Cauchy sequence in $M$.\\
Then $\{x_n\}$ converges.
\end{thm}
\pause
\begin{proof}
\pause
Consider the range $X = \{x_1,x_2,\dots\}$ of $\{x_n\}$.\\
\pause
Since $\{x_n\}$ is Cauchy, $X$ is bounded.\\
\pause
We consider two cases:
\begin{itemize}
\pause
\item when $X$ is finite
\pause
\item when $X$ is infinite
\end{itemize}
\end{proof}
\end{frame}

\begin{frame}{Cauchy sequence}
\begin{proof}
\textbf{Case I:} Assume $X$ is finite.\\
\pause
Then
$$\{d(x,y): x,y\in X,\ x\neq y\}$$
is a finite set of positive values.\\
\pause
Therefore it has a minimum $\epsilon > 0$.\\
\pause
Since $\{x_n\}$ is Cauchy, there exists $N\in\mathbb{Z}_+$ with $d(x_m,x_n) < \epsilon$ for all $m,n \geq N$.\\
\pause
This implies $x_n = x_N$ for all $n\geq N$.\\
\pause
Hence $\lim_{n\rightarrow\infty} x_n = x_N$.
\end{proof}
\end{frame}

\begin{frame}{Cauchy sequence}
\begin{proof}
\textbf{Case II:} Assume $X$ is infinite.\\
\pause
Then by the Bolzano-Weierstrass Theorem, $X$ has an accumulation point $L\in M$.\\
\pause
Let $\epsilon > 0$.\\
\pause
Since $\{x_n\}$ is Cauchy, we can choose $N\in\mathbb{Z}_+$ with $d(x_m,x_n) < \epsilon/2$ for all $m,n\geq N$.\\
\pause
Moreover, the ball $B(L,\epsilon /2)$ contains infinitely many points of $X$, so we can choose $\ell\geq N$ with $x_\ell\in B(L,\epsilon/2)$.\\
\pause
It follows that for all $n\geq N$,
$$d(x_n,L)\leq d(x_n,x_\ell) + d(x_\ell,L) \leq \epsilon/2 + \epsilon/2 = \epsilon.$$
\pause
Since $\epsilon > 0$ was arbitrary, this proves $\{x_n\}$ converges to $L$.
\end{proof}
\end{frame}

\begin{frame}{Example}
Consider the metric space $M=\mathbb{Q}$ with the Euclidean metric $d$.\\
\pause
Consider the sequence $\{x_n\}$ defined by $x_1 = 1$ and
$$x_{n+1} = x_n - \frac{x_n^2-2}{2x_n},\quad n\geq 1.$$
\pause
% $$x_{n+1} = \frac{x_n^2+2}{2x_n},\quad n\geq 1.$$
$$x_1 = 1,\quad x_2 = \frac{3}{2},\quad x_3 = \frac{17}{12},\dots$$
\begin{itemize}
\pause
\item \emph{tries} to converge to $\sqrt{2}$
\pause
\item however $\sqrt{2}\notin M$, so it doesn't converge!
\pause
\item somehow $M$ is "missing" some points
\end{itemize}
\end{frame}

\begin{frame}{Complete spaces}
\begin{defn}
A metric space $(M,d)$ is called \textbf{complete} if every Cauchy sequence in $M$ converges.\\
\pause
A subset $S\subseteq M$ is called \textbf{complete} if the subspace $(S,d)$ is a complete metric space.
\end{defn}
\begin{itemize}
\pause
\item $\mathbb R^n$ with the Euclidean metric is complete
\pause
\item $[0,1]$ with the Euclidean metric is complete
\pause
\item $(0,1)$ with the Euclidean metric is not complete
\end{itemize}
\end{frame}

\begin{frame}{Challenge!}
\begin{prob}
Show that the interval $(0,1)\subseteq\mathbb{R}$ with the Euclidean metric is not a complete metric space.
\end{prob}
\end{frame}




\end{document}


